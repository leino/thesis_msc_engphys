\subsection{Introduction and basic definitions}

In the last section, we showed how to implement MCTS (or more specifically: UCT), which can be used directly to play games in a probabilistic manner.
The plan is to try it out on many different games.
In order to do that, we have to generate the games themselves.
The goal of this chapter is to show how to do just that.

This section talks about computationally generating positional games, given some parameters of said games, which we will cover later.
Since we will use a tool called Nauty to generate games, and since Nauty talks about graphs, it is slightly more appropriate to talk about hyper-graphs in place of positional games.

\begin{definition}
  A \emph{hypergraph} is a set of \emph{vertices}, $V$, together with a list of \emph{hyperedges} $E$, which are simply non-empty subsets of $V$.
\end{definition}


A positional game is equivalent to a hypergraph, if we take the so-called winning sets to be the hyperedges.



\subsection{Nauty and hypergraphs}


Nauty is a tool to generate and work with graphs, and can be downloaded at http://cs.anu.edu.au/people/bdm/nauty/.

We are interested, particularly, in the tools \texttt{genbg} and \texttt{showg}.

\texttt{genbg} is used to generate (non-isomorphic) bipartite graphs in the very compact g6 format, and \texttt{showg} is used to turn these g6-formatted graphs into more human-readable form.


\subsection{Choosing the command line switches}
\label{sec:nautycommandline}

A bipartite graph corresponds to a hypergraph in the following manner.
Let us say that the two colors are red and blue. Then we can decide that the red vertices correspond exactly with the set of vertices in our hypergraph.
We can make the blue vertices correspond to hyperedges by defining the hyper edge for a given blue vertex as those red vertices which are connected to it.


In this section, we work out as an example all of the hypergraphs with 3 vertices and 2 hyperedges, using Nauty.
I will assume that Nauty has been installed and your current directory is wherever you installed Nauty or have \texttt{genbg} and \texttt{showg} available.

Here is some example output:

\begin{code}
   $ ./genbg 2 3 | ./showg


Graph 1, order 5.
  0 : ;
  1 : ;
  2 : ;
  3 : ;
  4 : ;

Graph 2, order 5.
  0 : 4;
  1 : ;
  2 : ;
  3 : ;
  4 : 0;

Graph 3, order 5.
  0 : 4;
  1 : 4;
  2 : ;
  3 : ;
  4 : 0 1;

Graph 4, order 5.
  0 : 3 4;
  1 : ;
  2 : ;
  3 : 0;
  4 : 0;

Graph 5, order 5.
  0 : 3;
  1 : 4;
  2 : ;
  3 : 0;
  4 : 1;

Graph 6, order 5.
  0 : 3 4;
  1 : 4;
  2 : ;
  3 : 0;
  4 : 0 1;

Graph 7, order 5.
  0 : 3 4;
  1 : 3 4;
  2 : ;
  3 : 0 1;
  4 : 0 1;

Graph 8, order 5.
  0 : 2 3 4;
  1 : ;
  2 : 0;
  3 : 0;
  4 : 0;

Graph 9, order 5.
  0 : 2 3 4;
  1 : 4;
  2 : 0;
  3 : 0;
  4 : 0 1;

Graph 10, order 5.
  0 : 2 4;
  1 : 3;
  2 : 0;
  3 : 1;
  4 : 0;

Graph 11, order 5.
  0 : 2 4;
  1 : 3 4;
  2 : 0;
  3 : 1;
  4 : 0 1;

Graph 12, order 5.
  0 : 2 3 4;
  1 : 3 4;
  2 : 0;
  3 : 0 1;
  4 : 0 1;

Graph 13, order 5.
  0 : 2 3 4;
  1 : 2 3 4;
  2 : 0 1;
  3 : 0 1;
  4 : 0 1;

\end{code}
The first three vertices are from the first color and correspond to the vertices of our would-be hypergraphs.
The remaining two vertices are of the second color and correspond to the would-be hyperedges.

Clearly, there are some issues to work out. Firstly, note that \texttt{Graph 1} does not give us an actual hypergraph: both hyperedges would be empty, under our interpretation, which we do not allow. The same criticism holds for \texttt{Graph 2} and \texttt{Graph 3}.

To get around this, we use the command line switch $\texttt{-dm:n}$ where the \texttt{m} and \texttt{n} are the minimum degree of the first and second class of vertices, respectively.

With something like \texttt{-d0:1} we are saying that the vertices of our hypergraph may be in no hyperedges, but each hyperedge must contain at least 1 vertex, i.e. must not be empty.


\begin{code}

  $ ./genbg 2 3 -d0:1 | ./showg



Graph 1, order 5.
  0 : 2 3 4;
  1 : ;
  2 : 0;
  3 : 0;
  4 : 0;

Graph 2, order 5.
  0 : 2 3 4;
  1 : 4;
  2 : 0;
  3 : 0;
  4 : 0 1;

Graph 3, order 5.
  0 : 2 4;
  1 : 3;
  2 : 0;
  3 : 1;
  4 : 0;

Graph 4, order 5.
  0 : 2 4;
  1 : 3 4;
  2 : 0;
  3 : 1;
  4 : 0 1;

Graph 5, order 5.
  0 : 2 3 4;
  1 : 3 4;
  2 : 0;
  3 : 0 1;
  4 : 0 1;

Graph 6, order 5.
  0 : 2 3 4;
  1 : 2 3 4;
  2 : 0 1;
  3 : 0 1;
  4 : 0 1;

\end{code}
There are still some issues. If we were to try to translate \texttt{Graph 1} to a hypergraph, we would get the same hyper edge three times. That is to say, we would end up with a hypergraph of only a single hyper edge. If we were interested in such a hypergraph, we would have just ran \texttt{./genbg 2 1 -d0:1} to begin with.

This issue is resolved with the command line switch \texttt{-z}, which makes sure that no two vertices in the second class can have the same neighborhood.

\begin{code}

  $ ./genbg 2 3 -z -d0:1 | ./showg

Graph 1, order 5.
  0 : 2 4;
  1 : 3 4;
  2 : 0;
  3 : 1;
  4 : 0 1;


\end{code}
So, in the end we have only a single hypergraph in this class. It has two vertices, each with it's own singleton hyperedge plus a hyperedge that contains both of the vertices.

Our hypothesis at this point is that a command such as \texttt{./genbg m n -z -d0:1} will generate all non-isomorphic hypergraphs of \texttt{m} vertices and \texttt{n} edges.

In other words we need to prove the following result.

\begin{theorem}

Let $B$ be the set of all bipartite graphs with $n$ vertices in the first class and $m$ vertices in the second class.
Suppose further that $B$ is such that all vertices in the second class have non-zero neighbourhoods, and that no two vertices in the second class have the same neighbourhood.

Then we get a one-to-one bijection to hypergraphs with $n$ vertices and $m$ hyperedges, by taking the vertices in the first class to be the vertices of the hypergraph, and the vertices in the second class to be hyperedges (defined by their neighbourhoods).

\end{theorem}
