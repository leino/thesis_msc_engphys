\subsection{Introduction}

In this section, we will introduce an algorithm called the minimax algorithm which can play any positional game (and more general games) in an optimal manner.
The natural structure on which algorithmic play takes place is the game tree, as described in section \ref{subsec:gametree}.
A playing algorithm can then be seen as a search algorithm on the game tree.

The minimax algorithm in it's most naive implementation will search trough the entire game tree. This not feasible for most games.
We therefore also describe a common optimization of the minimax algorithm, called alpha-beta pruning.


Most of this material is covered in \citep{aimodernapproach}, though with some of the propositions given as exercises and with more imperative-style pseudo code.

\subsection{Optimal play}

What do we mean by ``optimal play''?
In this section, we will give some pretty intuitive neccesary conditions.

Clearly, a player cannot play optimally if he squanders an opportunity to win.
More precisely; if current position is a win for the moving player, then he must make a choice which is also a win for him.
Furthermore, if the moving player does not have any opportunity to win in the given position (i.e. none of the leaf-nodes in the tree below the current node contains a winning node for the moving player), but if he does have an opportunity to tie, he must still have an opportunity to tie after he makes his move.

If a player can play as above, he will win if he can win, and if he can't win but he can tie, he will do that.

\subsection{The minimax algorithm}
\label{sec:minimax}

In this section, we will see how the considerations in the previous sections guide us to a pretty intuitive algorithm which leads to optimal play.
It is intuitively clear that it is always possible to play as described in the previous section.
The key in order to find an explicit algorithm is to extend the notion of First player win to be not just for leaf nodes.

In terms of the game tree, what might it mean for any given position, not just a leaf node, to be a First player win? The definition is inductive.
In the base case, i.e. we have a leaf node, the definition of winning node is clear from the rules of the particular game.
If we are not on a leaf node, we break the definition up into two cases:

\begin{definition}
\label{def:positionclasses}

\begin{itemize}
  \item Case 1: It is Firsts turn to move. In this case, the position is a First win it has a child which is a First win.

  \item Case 2: It is Seconds turn to move. In this case, the position is a First win if all of it's children have children which are First win.

\end{itemize}

\end{definition}
The notion of a position being a Second win is defined similarly.
It is easy to see that if a position is a First win, it cannot be a Second win, and vice-versa.
This does not mean that a position must be either a First win or a Second win; and we call such positions Neither win positions.
We have now defined a kind of coloring of any game tree, with three different colors: First win, Second win and Neither win.
Instead saying ``the color of a node'', it is more natural to speak of ``the \emph{winner} of a node'', which can be either First, Second or Neither.

The following two propositions follow directly from the definition of optimal play:

\begin{proposition}
If the game is in a First win position and First plays optimally, then the path in the tree represented by a play will consist entirely of First win nodes.
(And similarly for Second.)
\end{proposition}

\begin{proposition}
If the game is in a Neither win position and First plays optimally, then the play path does not contain nodes where Second wins.
(And similarly for Second.)
\end{proposition}
These two propositions are summed up in \citep{aimodernapproach} (page 197, exercise 5.7).
Note that, in the second proposition, the play path might also end with a sequence of nodes of First win color unless Second also plays optimally.

Finally, as a simple corollary of the above two propositions:

\begin{proposition}
If both players play optimally, then the nodes in the play path will all have the same winner (First, Second or Neither).
\end{proposition}
From the above results, we have the following nice re-characterization of definition \ref{def:positionclasses}.

\begin{theorem}
The winner for a given node is the same as the winner of the leaf node (end position) which results when both players play optimally.
\end{theorem}
To get something a bit more operational, we define an ordering on the set of colors:

\begin{equation}
\label{eq:winnerordering}
  \text{Second win} < \text{Neither win} < \text{First win}
\end{equation}
It is clear that First plays optimally if he at all times makes choices which are \emph{maximal} with respect to the above ordering.
Similarly, Second plays optimally if he at all times makes choices which are \emph{minimal} with respect to the same ordering.

Thus, to find the winner of a position in which it's Firsts time to move, we take the maximum of it's children, according to the above ordering.
To find the winner of a position in which it's Seconts time to move, we take the minimum.
This leads to the following recursive definition for the winner of a node, which can me used to play optimally:

\begin{code}
  winner position =
    | terminal position       = terminalWinner position
    | turn position == First  = max $ map winner $ choices position
    | otherwise               = min $ map winner $ choices position
\end{code}
The above is a taste of Haskell code.
The \texttt{min} and \texttt{max} functions know about the ordering defined in equation \ref{eq:winnerordering}.
Since we know the rules of the game, the functions \texttt{turn}, \texttt{choices} and \texttt{terminal} are assumed to be given.
(In turn, they tell who's turn it is, what his choices are and whether the position is terminal, given any position of the game.)

The code breaks the definition of \texttt{winner} up into three cases.
In the first case, where the position is terminal, we trivially know the winner from the definition of the game, so \texttt{terminalWinner} is assumed to be given.
In the second case, where the position is not terminal and it is Firsts turn to play, the winner is the maximum of \texttt{winner} mapped over the choices available from the given position.
In the final case, we know that the node is not terminal and it is not Firsts turn to move. Thus it is Seconds turn to move, and so the winner is the minimum of \texttt{winner} mapped over the choices available from the given position.

With the above function in hand, it is easy for either player to play optimally.
The actual code used to run the experiments later on is not quite this simple: as mentioned there's a common optimization which we will take advantage of.
Namely alpha-beta pruning.

\subsection {Alpha-Beta Pruning}

The idea for this optimization is quite simple.
Suppose that, as we are evaluating \texttt{max} function in the code segment from the previous section, and we run into a child with value First win.
We then know that we can stop searching because we cannot do any better than that.
Similarly, whilst evaluating the \texttt{min} function, if we run into a Second win, then we can also stop prematurely.

This is a specific case of alpha-beta pruning, and will allow us to disregard (or \emph{prune}) big parts of the game three as we search it.
There is a generalization where the leaf node can have a bigger set of values than just First win, Second win or Neither win, but we don't need it here.
The implementation is particularly easy in Haskell. We can even make it look exactly like the implementation above, but we need to take some special care when writing min and max so that they know what the absolute minimum is (Second win) and what the absolute maximum is (First win) and can therefore prune.

This means that we can find the maximum without actually computing the entire list of elements in the list (which would require searching the entire game tree), thanks to a feature of Haskell known as laziness.

Here is our version of \texttt{min} which is lazy and will consequently allow pruning:
\begin{code}
  prunedMin ws = 
    case find ((==) Only Second) ws of
      Nothing -> min ws
      _ -> Only Second
\end{code}
The implementation for prunedMax is similar:
\begin{code}
  prunedMax ws = 
    case find ((==) Only First) ws of
      Nothing -> max ws
      _ -> Only First
\end{code}
Now we can rewrite our pruned minimax algorithm so that it is very similar to the one in the previous section:
\begin{code}
  winner position
    | terminal position       = terminalWinner position
    | turn position == First  = prunedMax $ map winner $ choices play
    | otherwise               = prunedMin $ map winner $ choices play
\end{code}
Again, the winner function is the interesting part. If one has a winner function, it is easy to fill in the details required to derive a completely generic strategy.

In table \ref{tab:ex_alphabeta}, there is an example of the part of a game tree seen by minimax with alpha-beta pruning.
(Note that the root node is not an empty board, but instead a position in the middle of a play. This is because the tree would be very large if it started with an empty board.)
The gray squares represent positions that do not need to be considered, thanks to alpha-beta pruning.
The white squares represent positions which are not part of the game tree, but are included in the picture for visual balance.
\begin{center}
\def\arraystretch{5.5}
\begin{table}
\def\svgwidth{\columnwidth} \input{alpha-beta_example.pdf_tex}
\caption{An example of alpha-beta pruning}
\label{tab:ex_alphabeta}
\end{table}
\end{center}
