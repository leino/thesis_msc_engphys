\subsection {Overview}

In this section, we give an overview of the overall structure of the code and the \texttt{PoGa} library.


Some parts of the \texttt{PoGa} library, such as the implementations of the games Tic-Tac-Toe and Hex, are not relevant and are thus not brought up here.
The same goes for the entire \texttt{Graphics} submodule, which implements support for a human player for the cases of Tic-Tac-Toe and Hex.

\subsection {Positional game library}

In this section, we introduce the \texttt{PoGa} library.
One part of \texttt{PoGa} is mainly an interface used for representing and playing positional games.
However, \texttt{PoGa} also provides the two strategies discussed earlier: minimax with alpha-beta pruning, and the UCT variant of MCTS.

\texttt{PoGa} also contains implementations of arbitrarily sized ($m \times n$) Tic-Tac-Toe and Hex.

\subsubsection {Game.hs}

This module defines the \texttt{Position} type-class, which is the central interface in the \texttt{PoGa} module.
It also defines the notion of a \texttt{Strategy} and how to play two strategies against each other in \texttt{playGame}.

Note that a strategy is just a move function, but that the result of the move is a monadic position value.
If we use the \texttt{IO} monad, we can represent a human player as a strategy. If we use a monad in the \texttt{MonadRandom} type-class, we can represent a random strategy. (Though, typically not completely random, of course.)

\begin{code}
test
\end{code}

\subsubsection{SetGame.hs}

A typeclass works well as a general interface, but we need to have a concrete data-type to represent our games.

For many purposes, the following one works quite well. When it doesn't, a user of the library can always define his own data-type and use most of the functionality provided by the \texttt{PoGa} library, as long as the data-type is an instance of the \texttt{Position} typeclass, as described above.

\begin{code}
test
\end{code}

\subsubsection {Strategies.hs}

In this section, we list the strategy submodule of \texttt{PoGa}.
It contains implementations of minimax with alpha-beta pruning as well as UCT.
The implementations of these were discussed in less detail in sections \ref{sec:alpha_beta} and \ref{sec:uct}, respectively.


As mentioned, strategies take a position to a monadic position, where the monad typeically is in \texttt{MonadRandom} or is the \texttt{IO} monad.

A strategy representing a human player would require some kind of user interface, which is not provided here.
It is provided in the module named \texttt{GraphicalGame}, and the required user interfaces are specified for the cases of arbitrarily sized Tic-Tac-Toe and Hex games.

Since it is not an important part of this paper, that entire module is left out of this section for space reasons.
However, the interested reader can download them from the repository (TODO: insert repository address here).

\begin{code}
test
\end{code}
