\subsection{What is a combinatorial game?}

Game theory in general is a large field.
To give an idea about what kinds of games people have studied, here are two orthogonal categorizations.

Firstly there is the \emph {chance versus skill} aspect. There are games of pure chance and no skill, games of some chance and some skill, and games of pure skill.
Combinatorial games are games of pure skill and no chance, and Poker is a game of some chance and some skill.

Secondly, there are games of \emph{complete information} and games of \emph{incomplete information}.
Poker is a game of incomplete information, and combonatorial games are games of complete information.

We will be concerned only with combinatorial games played by two players.

\subsubsection{Examples of combinatorial games}

The game of \emph{Nim} is a good example of a combinatorial game.
Nim is a game for two players, and can be played using nothing more than a bunch of beans (or other small objects).
The game is played by arranging the beans in a number of heaps.
The players take turns performing moves. A move is made by selecting a heap, and then taking a number of beans from that heap only. The winner is the last person to take a bean.

Another very popular combinatorial game is \emph{checkers}. (Also known as \emph{draughts}.)
Checkers is played by two players on a checkerboard of some size, say 8-by-8 or 12-by-12 squares.
Each player starts out with pieces of a given color in a certain pattern on his side of the board.
The players then take turns performing alternate moves on his own pieces.
A move consists in selecting a piece of ones own color, and moving it to an adjacent location.
If the opponent has a piece in such a location, it may be captured by jumping over it to the next adjacent location in the same direction, in which case it is removed from the board. If a move captured an opponent piece, it may be extended in case it can capture more of the opponents pieces in the same way, in a subsequent move. A piece may also become a so-called ``king'', in which case it can move in more ways.

\subsection{Advances in combinatorial game theory}

For a given combinatorial game, it is quite easy to convince oneself that it would be possible, in principle, to exhaustively enumerate every single possible play of that game.
In a few of those plays, one player may act ideally in all situations, and in even fewer, both players will act ideally in all situations.

However, playing by investigating all possible plays will quickly introduce a player to the concept of \emph{combinatorial chaos} -- the number of possible ways of playing the game, though finite, can be enormous.
Therefore it is natural to wonder if one can find an algorithm, with reasonable computational complexity, to play a given game perfectly.

Several results along these lines are presented in \citep{demaine_hearn08}. For instance: on page 10 we learn that checkers has been determined to be a draw if both players play perfectly, but that playing perfectly is a hard computational problem. (PSPACE-hard.)

\subsubsection{Nim-like games and Sprauge-Grundy theory}

It should be mentioned that, in spite of combinatorial chaos, some elegant results have been found for a certain class of games -- so-called ``Nim-like'' games.

TODO: write about these results

\subsubsection{Important books}

TODO: reference ''Winning ways, by Conway''
TODO: reference ``Beck -- Tic-Tac-Toe theory''
