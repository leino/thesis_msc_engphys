In this section, we go trough, in a tutorial-like fashion, various interesting queries that one could make into the database.

It is assumed that you have installed SQLite.

\subsection{Overall database structure queries}

This section presents various ways of querying the overall structure of the database.
Feel free to experiment and combine these various queries, as long as you follow the rules of the SQL language implementation in SQLite.

\subsubsection{Query: Vertices and edges}

Suppose that we are interested in finding out, roughly, the structure of the table \texttt{mctsvsperfect} in our database.
We might first be interested in knowing which ``classes'' of hypergraphs are in the table, where by ``class'' we mean how many vertices and edges.



\begin{code}

sqlite> SELECT DISTINCT numvertices,numedges
        FROM mctsvsperfect NATURAL JOIN hypergraphs;

...

\end{code}

The output is supressed here, since it is not ordered.
If we want ordered results, we extend the previous query a little bit:


\begin{code}

sqlite> SELECT DISTINCT numvertices,numedges
        FROM mctsvsperfect NATURAL JOIN hypergraphs
        ORDER BY numvertices, numedges;

2|1
3|1
3|2
3|3
3|4
4|1
4|2
4|3
4|4
4|5
4|6
4|7
4|8
4|9
4|10
5|1
5|2
5|3
5|4
5|5
5|6
5|7
5|8
5|9
5|10
5|11
5|12
5|13
5|14
5|15
5|16
5|17
5|18
5|19
5|20
6|1
6|2
6|3
6|4
6|5
6|6
6|7
6|8
6|9
6|10
6|11
6|12
6|13

\end{code}





