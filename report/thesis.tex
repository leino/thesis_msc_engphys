\documentclass[12pt]{article}
\usepackage{amsmath}
\usepackage{geometry}
\usepackage{amsfonts}
\usepackage{amssymb}
\usepackage{natbib}
\usepackage{verbatim}
\usepackage{listings}
\usepackage{array}
\usepackage{alltt}
\usepackage{lscape}
\usepackage{enumerate}
\usepackage[utf8]{inputenc}
\usepackage{fancyvrb}
\usepackage{tikz}
\usepackage{multirow}
\DefineVerbatimEnvironment{code}{Verbatim}{fontsize=\small}
\DefineVerbatimEnvironment{datalisting}{Verbatim}{fontsize=\scriptsize}


\begin{document}
\lstset{language=Haskell,basicstyle=\ttfamily\small,breaklines=true}

\newtheorem{theorem}{Theorem}[section]
\newtheorem{lemma}[theorem]{Lemma}
\newtheorem{proposition}[theorem]{Proposition}
\newtheorem{definition}[theorem]{Definition}
\newtheorem{corollary}[theorem]{Corollary}
\newtheorem{conjecture}[theorem]{Conjecture}

\newenvironment{proof}[1][Proof]{\begin{trivlist}
\item[\hskip \labelsep {\bfseries #1}]}{\end{trivlist}}

% we want our definitions numbered!
%\newenvironment{definition}[1][Definition]{\begin{trivlist}
%\item[\hskip \labelsep {\bfseries #1}]}{\end{trivlist}}

\newenvironment{notation}[1][Notation]{\begin{trivlist}
\item[\hskip \labelsep {\bfseries #1}]}{\end{trivlist}}

\newenvironment{example}[1][Example]{\begin{trivlist}
\item[\hskip \labelsep {\bfseries #1}]}{\end{trivlist}}

\newenvironment{remark}[1][Remark]{\begin{trivlist}
\item[\hskip \labelsep {\bfseries #1}]}{\end{trivlist}}

%%%%%%%%%%%%%%%%%%%%%%%%%%%%%%%
% some custom commands
%%%%%%%%%%%%%%%%%%%%%%%%%%%%%%%

\newcommand{\qed}{\nobreak \ifvmode \relax \else      % qed sign used to mark end of proofs
      \ifdim\lastskip<1.5em \hskip-\lastskip         
      \hskip1.5em plus0em minus0.5em \fi \nobreak     
      \vrule height0.75em width0.5em depth0.25em\fi}  
\newcommand{\fg}{\mathfrak M}                         % set of fin. gen. modules

\newcommand{\isomarrow}{\overset\sim\to}              % isomorphism arrow

%%%%%%%%%%%%%%%%%%%%%%%%%%%%%%%

\title{Exact and Monte-Carlo algorithms for combinatorial games}
\date{February 24, 2014}
\author{Anders Leino}

\maketitle

\begin{abstract}
This thesis concerns combinatorial games and algorithms that can be used to play them.
Basic definitions and results about combinatorial games are covered, and an implementation of the minimax algorithm with alpha-beta pruning is presented.
Following this, we give a description and implementation of the common UCT variant of MCTS (Monte-Carlo tree search).
Then, a framework for testing the behavior of UCT as first player, at various numbers of iterations ~(namely 2,7, \dots 27), versus minimax as second player, is described.
Finally, we present the results obtained by applying this framework to the 2.2 million smallest non-trivial positional games having winning sets of size either 2 or 3.
It is seen that on almost all different classifications of the games studied, UCT converges quickly to near-perfect play.
\end{abstract}

\tableofcontents

\section{Introduction}
\label{sec:introduction}
\subsection{What is a combinatorial game?}

Game theory in general is a large field.
To give an idea about what kinds of games people have studied, here are two orthogonal categorizations.

Firstly there is the \emph {chance versus skill} aspect. There are games of pure chance and no skill, games of some chance and some skill, and games of pure skill.
Combinatorial games are games of pure skill and no chance, and Poker is a game of some chance and some skill.

Secondly, there are games of \emph{complete information} and games of \emph{incomplete information}.
Poker is a game of incomplete information, and combonatorial games are games of complete information.

We will be concerned only with combinatorial games played by two players.

\subsubsection{Examples of combinatorial games}

The game of \emph{Nim} is a good example of a combinatorial game.
Nim is a game for two players, and can be played using nothing more than a bunch of beans (or other small objects).
The game is played by arranging the beans in a number of heaps.
The players take turns performing moves. A move is made by selecting a heap, and then taking a number of beans from that heap only. The winner is the last person to take a bean.

Another very popular combinatorial game is \emph{checkers}. (Also known as \emph{draughts}.)
Checkers is played by two players on a checkerboard of some size, say 8-by-8 or 12-by-12 squares.
Each player starts out with pieces of a given color in a certain pattern on his side of the board.
The players then take turns performing alternate moves on his own pieces.
A move consists in selecting a piece of ones own color, and moving it to an adjacent location.
If the opponent has a piece in such a location, it may be captured by jumping over it to the next adjacent location in the same direction, in which case it is removed from the board. If a move captured an opponent piece, it may be extended in case it can capture more of the opponents pieces in the same way, in a subsequent move. A piece may also become a so-called ``king'', in which case it can move in more ways.

\subsection{Advances in combinatorial game theory}

For a given combinatorial game, it is quite easy to convince oneself that it would be possible, in principle, to exhaustively enumerate every single possible play of that game.
In a few of those plays, one player may act ideally in all situations, and in even fewer, both players will act ideally in all situations.

However, playing by investigating all possible plays will quickly introduce a player to the concept of \emph{combinatorial chaos} -- the number of possible ways of playing the game, though finite, can be enormous.
Therefore it is natural to wonder if one can find an algorithm, with reasonable computational complexity, to play a given game perfectly.

Several results along these lines are presented in \citep{demaine_hearn08}. For instance: on page 10 we learn that checkers has been determined to be a draw if both players play perfectly, but that playing perfectly is a hard computational problem. (PSPACE-hard.)

\subsubsection{Nim-like games and Sprauge-Grundy theory}

It should be mentioned that, in spite of combinatorial chaos, some elegant results have been found for a certain class of games -- so-called ``Nim-like'' games.

TODO: write about these results

\subsubsection{Important books}

TODO: reference ''Winning ways, by Conway''
TODO: reference ``Beck -- Tic-Tac-Toe theory''


\section{Positional games}
\label{sec:positional_games}
\subsection{Definitions}

To define a \emph{positional game}, or \emph{strong game}, or simply \emph{game} from now on, we need a few things.

First of all, we need a ``board'', $V$ and a collection of ``winning subsets'', $\mathcal F \subset \mathcal P(V)$.

The tuple $(V,\mathcal F)$ constitutes a \emph{hypergraph}.
The set $V$ is sometimes called the \emph{vertices} and the set $\mathcal F$ is sometimes called the \emph{hyperedges} of the hypergraph $(V,\mathcal F)$.

\begin{remark}
We often refer to $(V,\mathcal F)$ as the game.
\end{remark}

The idea is that two players, called $\emph{First}$ and $\emph{Second}$ take turns coloring uncolored vertices of the board.

Initially, the entire board, $V$ starts out with all vertices uncolored.

The object of the game is to be the first to color an entire winning set.
The players are named as they are because player First has the benefit of getting the first move.

Note that a vertex which has already been colored cannot be colored again.

The word \emph{play} is meant to represent an instance of a correctly played game from start to finish.
Roughly speaking, this means a sequence of partial colorings of $V$.
(TODO: is partial and complete (see below) standard?)

A given point of the play is called a \emph{position} of the board.
More precisely, a position is a (partial or complete) two-coloring of $V$.

\begin{remark}
Note that a position may not necessarily be attainable when playing by the rules; for instance, I can color one vertex Second without any vertex colored First, which is obviously not a position that can arise in the course of a game played by the rules, but nevertheless a position by definition.
\end{remark}

A \emph{draw} happens when the board is fully occupied, yet neither player occupies completely a winning set.

\subsubsection{Reverse games}

The \emph{reverse} of a given game is obtained if the desired outcome is to avoid occupying completely the winning sets from $\mathcal F$.

(TODO: is $reverse(game)$ an operation? i.e. is it possible to define a game from a reverse game? i.e. is it possible to define winning sets for $reverse(game)$, in terms of the winning sets of game? If not, then give counter-example!)



\subsubsection{Weak games}

In the above definition of a a game, both players (First and Second), strive to occupy the same winning sets, given by $\mathcal F$. A player might be interested in settling for a draw. (For instance, if Second knows that he cannot win.) 

Thus, we have two players: one player is the \emph{Maker}, and one is the \emph{Breaker}.

We say that Maker wins if he manages to occupy completely one of the winning sets in $\mathcal F$, and Breaker wins if he manages to prevent maker.

Note that the notion of who is ``first to win'' is moot: a player either wins or doesn't win.
Note further that a draw is impossible in a Maker-Breaker game.

This ``game'' is called a the \emph{weak version} of the original game, or the corresponding \emph{Maker-Breaker} game.
Note however, that technically this isn't a game at all, according to the above definition: Maker and Breaker have different winning sets! Nevertheless, this is obviously a useful notion.

\subsubsection{Reverse weak games}

The notion of a \emph{reverse weak game} should now be intuitively clear.
The idea is to start with a game, get the corresponding weak game and then reverse that.
However, since a weak game is technically a not a game, we should give an explicit definition of what reverse means in reference to a weak game.

Suppose we have a game with the hypergraph $(V,\mathcal F)$.
The corresponding weak game as a Maker and a Breaker.
The reverse weak game has a player trying to avoid making, and a player trying to avoid breaking.
That is: we have an \emph{Avoider} and an \emph{Enforcer}.

Avoider tries to avoid occupying completely a winnig set from $\mathcal F$, and Enforcer tries to prevent Avoider from doing so, that is, tries to enforce Avoider into occupying completely a winning set.

(TODO: Is a reverse weak game a weak game? if not, then give a counter-example)

\subsection{Library interface}

(TODO: This section)


\subsection{The game tree}

A good conceptual tool when reasoning about positional games is the so-called game tree, corresponding to a given game.

The root node of the game tree is the starting position of the game.
That is to say, the empty board.

The children of the root node is all possible positions of the board after First has made his move.
The children of \emph{those} nodes are the position after Second has made his move, and so on.

The game tree clearly finite, but can be very large.

TODO: Insert example of game tree here


\section{Optimal play}
\label{sec:optimal_play}
\subsection{Introduction}

In this section, we will introduce an algorithm called the minimax algorithm which can play any positional game (and more general games) in an optimal manner.

Naturally, any such algorithm will suffer from performance issues.
Nevertheless, there is one common optimization called alpha-beta pruning, which we will also describe.
The natural structure on which algorithmic play takes place is the game tree, as described in section \ref{subsec:gametree}.
A playing algorithm can then be seen as a search algorithm on the game tree.


Most of this material is covered in \citep{aimodernapproach}, though with some of the propositions given as exercises and with more imperative pseudo code.

\subsection{Optimal play}

What do we mean by ``optimal play''?
In this section, we will give some pretty intuitive neccesary conditions.

Clearly, a player cannot play optimally if he squanders an opportunity to win.
More precisely; if current position is a win for the moving player, then he must make a choice which is also a win for him.
Furthermore, if the moving player does not have any opportunity to win in the given position (i.e. none of the leaf-nodes in the tree below the current node contains a winning node for the moving player), but if he does have an opportunity to tie, he must still have an opportunity to tie after he makes his move.

If a player can play as above, he will win if he can win, and if he can't win but he can tie, he will do that.

\subsection{The minimax algorithm}
\label{sec:minimax}

In this section, we will see how the considerations in the previous sections guide us to a pretty intuitive algorithm which leads to optimal play.
It is intuitively clear that it is always possible to play as described in the previous section.
The key in order to find an explicit algorithm is to extend the notion of First player win to be not just for leaf nodes.

In terms of the game tree, what might it mean for any given position, not just a leaf node, to be a First player win? The definition is inductive.
In the base case, i.e. we have a leaf node, the definition of winning node is clear from the rules of the particular game.
If we are not on a leaf node, we break the definition up into two cases:

\begin{definition}
\label{def:positionclasses}

\begin{itemize}
  \item Case 1: It is Firsts turn to move. In this case, the position is a First win it has a child which is a First win.

  \item Case 2: It is Seconds turn to move. In this case, the position is a First win if all of it's children have children which are First win.

\end{itemize}

\end{definition}
The notion of a position being a Second win is defined similarly.
It is easy to see that if a position is a First win, it cannot be a Second win, and vice-versa.
This does not mean that a position must be either a First win or a Second win; and we call such positions Neither win positions.
We have now defined a kind of coloring of any game tree, with three different colors: First win, Second win and Neither win.
Instead saying ``the color of a node'', it is more natural to speak of ``the \emph{winner} of a node'', which can be either First, Second or Neither.

The following two propositions follow directly from the definition of optimal play:

\begin{proposition}
If the game is in a First win position and First plays optimally, then the path in the tree represented by a play will consist entirely of First win nodes.
(And similarly for Second.)
\end{proposition}

\begin{proposition}
If the game is in a Neither win position and First plays optimally, then the play path does not contain nodes where Second wins.
(And similarly for Second.)
\end{proposition}
These two propositions are summed up in \citep{aimodernapproach} (page 197, exercise 5.7).
Note that, in the second proposition, the play path might also end with a sequence of nodes of First win color unless Second also plays optimally.

Finally, as a simple corollary of the above two propositions:

\begin{proposition}
If both players play optimally, then the nodes in the play path will all have the same winner (First, Second or Neither).
\end{proposition}
From the above results, we have the following nice re-characterization of definition \ref{def:positionclasses}.

\begin{theorem}
The winner for a given node is the same as the winner of the leaf node (end position) which results when both players play optimally.
\end{theorem}
To get something a bit more operational, we define an ordering on the set of colors:

\begin{equation}
\label{eq:winnerordering}
  \text{Second win} < \text{Neither win} < \text{First win}
\end{equation}
It is clear that First plays optimally if he at all times makes choices which are \emph{maximal} with respect to the above ordering.
Similarly, Second plays optimally if he at all times makes choices which are \emph{minimal} with respect to the same ordering.

Thus, to find the winner of a position in which it's Firsts time to move, we take the maximum of it's children, according to the above ordering.
To find the winner of a position in which it's Seconts time to move, we take the minimum.
This leads to the following recursive definition for the winner of a node, which can me used to play optimally:

\begin{code}
  winner position =
    | terminal position       = terminalWinner position
    | turn position == First  = max $ map winner $ choices position
    | otherwise               = min $ map winner $ choices position
\end{code}
The above is a taste of Haskell code.
The \texttt{min} and \texttt{max} functions know about the ordering defined in equation \ref{eq:winnerordering}.
Since we know the rules of the game, the functions \texttt{turn}, \texttt{choices} and \texttt{terminal} are assumed to be given.
(In turn, they tell who's turn it is, what his choices are and whether the position is terminal, given any position of the game.)

The code breaks the definition of \texttt{winner} up into three cases.
In the first case, where the position is terminal, we trivially know the winner from the definition of the game, so \texttt{terminalWinner} is assumed to be given.
In the second case, where the position is not terminal and it is Firsts turn to play, the winner is the maximum of \texttt{winner} mapped over the choices available from the given position.
In the final case, we know that the node is not terminal and it is not Firsts turn to move. Thus it is Seconds turn to move, and so the winner is the minimum of \texttt{winner} mapped over the choices available from the given position.

With the above function in hand, it is easy for either player to play optimally.
The actual code used to run the experiments later on is not quite this simple: as mentioned there's a common optimization which we will take advantage of.
Namely alpha-beta pruning.

\subsection {Alpha-Beta Pruning}

The idea for this optimization is quite simple.
Suppose that, as we are evaluating \texttt{max} function in the code segment from the previous section, and we run into a child with value First win.
We then know that we can stop searching because we cannot do any better than that.
Similarly, whilst evaluating the \texttt{min} function, if we run into a Second win, then we can also stop prematurely.

This is a specific case of alpha-beta pruning, and will allow us to disregard (or \emph{prune}) big parts of the game three as we search it.
There is a generalization where the leaf node can have a bigger set of values than just First win, Second win or Neither win, but we don't need it here.
The implementation is particularly easy in Haskell. We can even make it look exactly like the implementation above, but we need to take some special care when writing min and max so that they know what the absolute minimum is (Second win) and what the absolute maximum is (First win) and can therefore prune.

This means that we can find the maximum without actually computing the entire list of elements in the list (which would require searching the entire game tree), thanks to a feature of Haskell known as laziness.

Here is our version of \texttt{min} which is lazy and will consequently allow pruning:
\begin{code}
  prunedMin ws = 
    case find ((==) Only Second) ws of
      Nothing -> min ws
      _ -> Only Second
\end{code}
The implementation for prunedMax is similar:
\begin{code}
  prunedMax ws = 
    case find ((==) Only First) ws of
      Nothing -> max ws
      _ -> Only First
\end{code}
Now we can rewrite our pruned minimax algorithm so that it is very similar to the one in the previous section:
\begin{code}
  winner position
    | terminal position       = terminalWinner position
    | turn position == First  = prunedMax $ map winner $ choices play
    | otherwise               = prunedMin $ map winner $ choices play
\end{code}
Again, the winner function is the interesting part. If one has a winner function, it is easy to fill in the details required to derive a completely generic strategy.


\section{MCTS}
\label{sec:mcts}
\subsection {Introduction}

Monte-Carlo tree search (MCTS) has become an umbrella term for a class of related algorithms for searching trees probabilistically.
This applies directly to games if we decide to search the game tree.

In this section, we will introduce an MCTS algorithm known as UCT.
We will mostly follow the exposition in \citep{mcts_survey12}, chapter 3.

\subsection {MCTS in general}

MCTS studies the game tree as follows.
It keeps a record of a subtree of the game-tree containing the nodes that the algorithm has visited so far.
It also keeps some extra information about each node, which is supposed to represent an approximation of the ``value'' of that node.
The idea is to somehow find a good \emph{expandable node} (meaning that it has unvisited children) in the visited part of the game tree, and then to make an excursion from that node, which means doing a quicker kind of search from the node, in order to estimate the value of the node. The information gleaned from this excursion will then contribute to the algorithms knowledge of the game tree.

It is assumed that we have a (reasonably efficient) function that lets us determine the value of a leaf node.
Here is a sketch of the steps that will make up our algorithm:

\begin{itemize}
\item \emph{Selection}: find a suitable expandable explored node by repeatedly applying a \emph{selection function}, and select one of its child nodes.
\item \emph{Exploration}: run a simulation from the newly found child node and return a \emph{score}.
\item \emph{Backpropagation}: use the score found in the previous step to update the visited tree in an appropriate way.
\end{itemize}
These steps are iterated a number of times in order to make a single move. Each iteration yields a more complete and refined knowledge of the game tree, thanks to the backpropagation step. In order to subsequently make a move, a single application of the selection function is made.
Note that there are variants of this algorithm which expand and explore multiple nodes instead of just one, but the principle is the same otherwise.
Note also that this algorithm is far from complete. There are various appropriate ways of performing each of these steps, depending on the situation.
The next section describes one of the possibilities: the UCT (Upper Confidence bounds for Trees) algorithm.

\subsection{The UCT algorithm}
\label{sec:uct}

In this section, we fill in each of the steps outlined in the previous section, for the special case of the UCT algorithm.

Each node $v$ in the explored part of the game tree has an attached score, which is just a real-valued number, say $S(v)$.

\subsubsection{The selection step}
\label{subsec:uct_selection_step}

Selection takes place in the explored part of the game tree, and can therefore use the score, $S$.
We repeatedly pick the ``best child'' of the current node, in the following sense.
If $v$ has children which have not been explored, then pick any of them as the best child.
If all children of $v$ have already been explored, then we pick a child, $v'$, which maximizes

\begin{equation}
\label{eq:uctnodevalue}
S(v') = \frac{Q(v')}{N(v')} + c\sqrt{\frac{2\ln{N(v)}}{N(v')}}
\end{equation} 
where $v'$ is a child of $v$, $N$ is the visit count and $Q$ is the accumulated score for a node (we will see later how to keep track of $Q$ and $N$, for a given node).
The parameter $c$ determines the amount of exploration. We will choose $c = 1 / \sqrt 2$ as per the comments in \citep[p. 9]{mcts_survey12}.

This selection process continues until we find either an unexplored node or run into a node without children (i.e. a leaf node), in which case we return that leaf node.

\subsubsection{The exploration step}
\label{sec:exploration_step_explanation}

When we have found a node using the selection step, we will explore that node, which will yield a score.
If we are ``exploring'' a leaf node, the outcome is just First, Second or Neither, according to what outcome the leaf node represents.
In case we are not exploring a leaf node, then we are exploring an unexplored non-leaf node, and then simply search randomly from that node until we run into a leaf node, which we know how to evaluate an outcome for.
If the node has not previously been explored, it will finally be marked as explored, and its visit count and score will be initialized as appropriate.

The result of the exploration step is a real-valued number -- a score.
The score depends on who moves in order to arrive at the explored node, as well as what the outcome of the random search was.
If First moves in order to arrive at the explored node, then a First, Second or Neither win outcome has a value of $1$, $-1$ or $0$, respectively. If Second moves to the explored node, these scores are negated.

\subsubsection{The backup step}
\label{sec:backup_step_explanation}

As explained, the exploration step yields a score which is either $1$, $-1$ or $0$.
In the backup step we go back up the way we came, all the way to the root node.
As we go, we increment the visit count, $N$.
We also update the cumulative score, $Q$, in the following way.
Going up the tree, we bring with us the score corresponding to the outcome of the exploration step.
However, in order to reflect the fact that the two players have opposite opinions of the outcome, we make sure to alternate its sign on each step. This alternated score is added to the cumulative score, $Q$.

\subsection{An example}
In this section, we give an example of the UCT algorithm described above.
To keep things simple, we do only a single iteration.
However, in order to get a non-trivial iteration, we assume that five iterations have already been done, and do iteration number six.
\begin{center}
\def\arraystretch{5.5}
\begin{table}
\begin{tabular}{l}
  \def\svgwidth{\columnwidth} \input{mcts_example_before.pdf_tex}
\end{tabular}
\caption{Before the iteration}
\label{tab:mcts_iteration_before}
\end{table}
\end{center}
In table \ref{tab:mcts_iteration_before}, we have our starting point.
We must first carry out the selection step. As can be seen from the figure, there are two cases: one with $N=1$ and $Q=-1$, and one with $N=1$ and $Q=1$.
Let the choices on the second row of table \ref{tab:mcts_iteration_before} be $v_1$, $v_2$, $v_3$ and $v_4$, from left to right.
Using expression \ref{eq:uctnodevalue} with $c = 1/\sqrt{2}$, we get $S(v_1) = \sqrt{\ln{5}} - 1$ for the first choice, and $S(v_2) = S(v_3) = S(v_4) = \sqrt{\ln{5}} + 1$ for the other three choices.
So $v_2$, $v_3$ and $v_4$ all maximize $S$.
Suppose we choose to explore $v_2$. Since this node has no explored children, we do not need to calculate \ref{eq:uctnodevalue} -- we simply select the first child, i.e. the fourth node from the left, on the third row.
This node is the result of the selection step.

Now it is time to perform the exploration step.
This is just a random search from the selected node, until we hit a leaf node.
As can be seen from table \ref{tab:mcts_iteration_before}, there are two possible outcomes: either the seventh node on the fourth row, or the eight node on the fifth row.
Suppose that the outcome is the latter.
This node is a win for First. Since we are exploring a node which First moves to, the numerical score of this outcome is $1$, as per the explanation in \ref{sec:exploration_step_explanation}.
Thus, the backup step starts with the score $1$, and so the explored node gets initialized with $N=1$ and $Q=$.
The backup step is not done yet: we should continue all the way up to the root node, remembering to alternate the sign and adjust the accumulated scores and visit counts as we go, as described in \ref{sec:backup_step_explanation}.
The complete result after the backup step is shown in table \ref{fig:mcts_iteration_after}.
\begin{center}
\def\arraystretch{5.5}
\begin{table}
\begin{tabular}{l}
  \def\svgwidth{\columnwidth} \input{mcts_example_after.pdf_tex}
\end{tabular}
\caption{After the iteration}
\label{fig:mcts_iteration_after}
\end{table}
\end{center}

\subsection{UCT implementation}

It is assumed that we have a function, \texttt{value}, which can only be applied to terminal values and which gives a real number representing the value of a given node from the point of view of the player in turn (i.e. the opponent of the player who made the previous move). The value is $1$ if the position is a winning position for the player in turn, $-1$ if it is a losing position, and $0$ otherwise.
We also assume that we have a function named \texttt{choices}, which we can apply to a node in order to find the set of possible choices of nodes that the player in turn could move to. Finally, we assume that our nodes may carry ``MCTS data'' of the following format.
\begin{lstlisting}[frame=single]
data MCTSNodeData = MCTSNodeData { visitCount :: Int,
                                   score :: Score }
\end{lstlisting}
A node only carries \texttt{MCTSNodeData} if it has been explored by the \texttt{explore} function, below.
The \texttt{Score} type is just a synonym for a real number type, like \texttt{Float}. The function \texttt{getMCTSData} will return the MCTS data for a given node if it has any, and we can set it by means of \texttt{setMCTSData}.

We begin by looking at the \texttt{recon} function, which denotes the score \emph{from the point of view of the player in turn}, of a random search from the given position.
\begin{minipage}{\linewidth}
\begin{lstlisting}[frame=single]
recon position
  | Game.terminal position = return $ value pos
  | otherwise = do
      c <- Random.fromList [(c,1)
                           | c <- Game.choices position]
      s <- recon c
      return $ -s
\end{lstlisting}
\end{minipage}
It reads as follows: in case the given position is terminal, then we can return the \texttt{value} of that position. Otherwise we select a random child, apply \texttt{recon} to it, and get a score, \texttt{s}, back.
Note that $\texttt{s}$ represents the score from the point of view of the opponent of the player in turn, since \texttt{recon} was applied to a child of \texttt{position}. Therefore we must negate the score before we return it.

Next, we'll look at the \texttt{explore} function, which is the core of the algorithm. The return value of the function is a tuple of a score together with the explored node. Just like \texttt{recon}, the score it returns is relative to the player who's turn it is in the given node. Apart from the node to explore, it also takes \texttt{cExp}, which is just the $c$ parameter from expression~\ref{eq:uctnodevalue}.

\begin{minipage}{\linewidth}
\begin{lstlisting}[frame=single]
explore cExp node = 
  case getMCTSData node of
    Nothing -> do
      s <- if Game.terminal node
           then return $ value node
           else recon node
      return ( s,
               setMCTSData node $
               MCTSNodeData {visitCount = 1,
                             score = s} )
    Just (MCTSNodeData {visitCount = vc,
                        score = sc}) -> do
      case Game.terminal node of
        True -> do
          let s = value node
          return ( s,
                   setMCTSData node $
                   MCTSNodeData {visitCount = vc + 1,
                                 score = sc + s} )
        False -> do
          let (c, cs) = popBestChild cExp node
          (s, c') <- explore cExp c
          let s' = negate s
              node' = setChoices node (c':cs) in
            return ( s',
                     setMCTSData node' $
                     MCTSNodeData {visitCount = vc + 1,
                                   score = sc + s'} )
\end{lstlisting}
\end{minipage}
At the top level, the function is split up into two cases -- either our node has no MCTS data, or it does.
If the node does not have MCTS data, we obtain a score, \texttt{s}, in either of two ways: using the \texttt{value} function if the node is terminal, or else by applying the \texttt{recon} to the node, i.e. doing a random search.
In either case we get a score, and so we can return the explored version of our node.

In case the node has MCTS data, i.e. is explored, we again have two sub-cases: terminal or not terminal.
If the explored node is terminal, we again use the value function to obtain a score, and use that to update the MCTS data for the node.
If the explored node is not terminal, then we select the best child as described in section~\ref{subsec:uct_selection_step}, explore that child recursively with another call to \texttt{explore} and use the result to return an updated node.

One important ingredient is the \texttt{compareChildren} function. It takes a node and two children, \texttt{a} and \texttt{b}, of the node, and returns an ordering, which is just a type for representing ``less than'', ``equal to'' or ``greater than''.
It allows us to sort a set of children, and therefore to write \texttt{popBestChild} and \texttt{findBestMove} (below), with relative ease.
The function is just a straight encoding of the rules mentioned in \ref{subsec:uct_selection_step}.
\begin{lstlisting}[frame=single]
compareChildren cExp node a b =
  case (getMCTSData a, getMCTSData b) of
    (Just aData, Just bData) ->
      let Just parentData = getMCTSData node in
      compare (reconScore parentData aData) (reconScore parentData bData)    
      where
        reconScore :: MCTSNodeData -> MCTSNodeData -> Score
        reconScore parentData childData =
          let (vcp, sp) = (visitCount $ parentData, score $ parentData)
              (vcc, sc) = (visitCount $ childData, score $ childData) in
          ( sc / (fromIntegral vcc) ) +
          cExp * sqrt (  2.0*(log $ fromIntegral vcp) / (fromIntegral vcc)  )
    (Nothing, Nothing) -> EQ
    (Just _, Nothing) -> LT
    (Nothing, Just _) -> GT
\end{lstlisting}
The function splits into two cases. In the case where both children have been explored (i.e. have got MCTS data), we use expression~\ref{eq:uctnodevalue} to produce two numbers which us an ordering in the usual way. In case neither child have been explored, they are equal. In case the first has been explored but not the second, then the first is less than the second, and the final case is just the reverse of this.

With the \texttt{explore} function and \texttt{compareChildren} in hand, we can easily write the final MCTS strategy. It needs two parameters: a number of iterations and a node to work on.

\begin{minipage}{\linewidth}
\begin{lstlisting}[frame=single]
mctsStrategy 0 node = do
  return $ findBestMove node
mctsStrategy numSteps node = do
  (_, node') <- explore cExp node
  mctsStrategy (numSteps-1) node'
  where
  cExp :: Double  
  cExp = 1.0 / (sqrt 2.0)
\end{lstlisting}
\end{minipage}
The base case of zero iterations is listed first. It uses the function \texttt{findBestMove} which determines the best move for the player in turn, according to the ordering implied by \texttt{compareChildern}. In the other case, one round of \texttt{explore} is executed, and the strategy is applied recursively to the resulting node.

There are still a few minor blanks left to fill in. For the complete code, see section~\ref{sec:code}.


\section{Generating games}
\label{sec:generating_games}
\subsection{Introduction and basic definitions}

In the last section, we showed how to implement MCTS (or more specifically: UCT), which can be used directly to play games in a probabilistic manner.
The plan is to try it out on many different games.
In order to do that, we have to generate the games themselves.
The goal of this chapter is to show how to do just that.

This section talks about computationally generating positional games, given some constraints which we will cover later.
Since we will use a tool called Nauty to generate games, and since Nauty talks about graphs, it is slightly more appropriate to talk about hyper-graphs in place of positional games.

\begin{definition}
  A \emph{hypergraph} is a set of \emph{vertices}, $V$, together with a set of \emph{hyperedges} $E$, which are non-empty subsets of $V$.
\end{definition}

A positional game is equivalent to a hypergraph, if we take the so-called winning sets to be the hyperedges.

\subsection{Nauty and hypergraphs}


Nauty is a tool to generate and work with graphs, and can be downloaded at \texttt{http://cs.anu.edu.au/people/bdm/nauty/}.
We are particularly interested in the tools \texttt{genbg} and \texttt{showg}.

\texttt{genbg} is used to generate (non-isomorphic) bipartite graphs. By default it will output the graphs in the very compact g6 format.
\texttt{showg} is used to turn these g6-formatted graphs into more human-readable form.

A bipartite graph corresponds to a hypergraph in the following manner.
Let us say that the two colors of the bipartite graph are red and blue. Then we can decide that the red vertices correspond exactly to the set of vertices in our hypergraph.
We can make the blue vertices correspond to hyperedges by defining the hyperedge for a given blue vertex as those red vertices which are connected to it.


\subsection{Choosing the command line switches}
\label{sec:nautycommandline}

In this section, we work out as an example all of the hypergraphs with 2 vertices and 3 hyperedges, using Nauty.
I will assume that Nauty has been installed and your current working directory contains the executables \texttt{genbg} and \texttt{showg}.

Here is the command to generate and display all non-isomorphic bipartite graphs with 2 vertices in the first class and 3 vertices in the second class.

\begin{code}
   $ ./genbg 2 3 | ./showg
\end{code}

The output (in tabular form here, for compactness) is as follows:

\begin{tabular}{|p{1.5in} | p{1.5in} | p{1.5in} |}
\begin{minipage}{1.5in}
\begin{datalisting}
Graph 1, order 5.
0 : ;
1 : ;
2 : ;
3 : ;
4 : ;

\end{datalisting}
\end{minipage}
&
\begin{minipage}{1.5in}
\begin{datalisting}
Graph 5, order 5.
0 : 3;
1 : 4;
2 : ;
3 : 0;
4 : 1;

\end{datalisting}
\end{minipage}
& 
\begin{minipage}{1.5in}
\begin{datalisting}
Graph 9, order 5.
0 : 2 3 4;
1 : 4;
2 : 0;
3 : 0;
4 : 0 1;

\end{datalisting}
\end{minipage}
\\
\begin{minipage}{1.5in}
\begin{datalisting}
Graph 2, order 5.
0 : 4;
1 : ;
2 : ;
3 : ;
4 : 0;

\end{datalisting}
\end{minipage}
&
\begin{minipage}{1.5in}
\begin{datalisting}
Graph 6, order 5.
0 : 3 4;
1 : 4;
2 : ;
3 : 0;
4 : 0 1;

\end{datalisting}
\end{minipage}
&
\begin{minipage}{1.5in}
\begin{datalisting}
Graph 10, order 5.
0 : 2 4;
1 : 3;
2 : 0;
3 : 1;
4 : 0;

\end{datalisting}
\end{minipage}
\\
\begin{minipage}{1.5in}
\begin{datalisting}
Graph 3, order 5.
0 : 4;
1 : 4;
2 : ;
3 : ;
4 : 0 1;

\end{datalisting}
\end{minipage}
&
\begin{minipage}{1.5in}
\begin{datalisting}
Graph 7, order 5.
0 : 3 4;
1 : 3 4;
2 : ;
3 : 0 1;
4 : 0 1;

\end{datalisting}
\end{minipage}
&
\begin{minipage}{1.5in}
\begin{datalisting}
Graph 11, order 5.
0 : 2 4;
1 : 3 4;
2 : 0;
3 : 1;
4 : 0 1;

\end{datalisting}
\end{minipage}
\\
\begin{minipage}{1.5in}
\begin{datalisting}
Graph 4, order 5.
0 : 3 4;
1 : ;
2 : ;
3 : 0;
4 : 0;

\end{datalisting}
\end{minipage}
&
\begin{minipage}{1.5in}
\begin{datalisting}
Graph 8, order 5.
0 : 2 3 4;
1 : ;
2 : 0;
3 : 0;
4 : 0;

\end{datalisting}
\end{minipage}
&
\begin{minipage}{1.5in}
\begin{datalisting}
Graph 12, order 5.
0 : 2 3 4;
1 : 3 4;
2 : 0;
3 : 0 1;
4 : 0 1;

\end{datalisting}
\end{minipage}
\\
&
&
\begin{minipage}{1.5in}
\begin{datalisting}
Graph 13, order 5.
0 : 2 3 4;
1 : 2 3 4;
2 : 0 1;
3 : 0 1;
4 : 0 1;

\end{datalisting}
\end{minipage}
\\
\end{tabular}


Each graph is displayed as five rows -- one row for each vertex.
The rows contain the index of the vertex, followed by a list of it's neighbours.
The first two rows correspond to the two vertices in the first class and thus correspond to the vertices of our would-be hypergraphs.
The remaining three rows correspond to the vertices of the second color, and thus correspond to the would-be hyperedges.
Clearly, there are some issues to work out. Firstly, note that \texttt{Graph 1} does not give us an actual hypergraph; both hyperedges would be empty, which we do not allow. The same criticism holds for \texttt{Graph 2} and \texttt{Graph 3}.

To get around this, we use the command line switch $\texttt{-dm:n}$ where the \texttt{m} and \texttt{n} are the minimum degree of the first and second class of vertices, respectively.

With \texttt{-d0:1} we are saying that the vertices of our hypergraph may be in no hyperedges, but each hyperedge must contain at least one vertex, i.e. must not be empty.


\begin{code}
  $ ./genbg 2 3 -d0:1 | ./showg
\end{code}
The output is:

\begin{tabular}{|c|c|c|}
\begin{minipage}{1.5in}
\begin{datalisting}
Graph 1, order 5.
  0 : 2 3 4;
  1 : ;
  2 : 0;
  3 : 0;
  4 : 0;

\end{datalisting}
\end{minipage}

&

\begin{minipage}{1.5in}
\begin{datalisting}
Graph 3, order 5.
  0 : 2 4;
  1 : 3;
  2 : 0;
  3 : 1;
  4 : 0;

\end{datalisting}
\end{minipage}

&

\begin{minipage}{1.5in}
\begin{datalisting}
Graph 5, order 5.
  0 : 2 3 4;
  1 : 3 4;
  2 : 0;
  3 : 0 1;
  4 : 0 1;

\end{datalisting}
\end{minipage}

\\

\begin{minipage}{1.5in}
\begin{datalisting}
Graph 2, order 5.
  0 : 2 3 4;
  1 : 4;
  2 : 0;
  3 : 0;
  4 : 0 1;

\end{datalisting}
\end{minipage}

&

\begin{minipage}{1.5in}
\begin{datalisting}
Graph 4, order 5.
  0 : 2 4;
  1 : 3 4;
  2 : 0;
  3 : 1;
  4 : 0 1;

\end{datalisting}
\end{minipage}

&

\begin{minipage}{1.5in}
\begin{datalisting}
Graph 6, order 5.
  0 : 2 3 4;
  1 : 2 3 4;
  2 : 0 1;
  3 : 0 1;
  4 : 0 1;
\end{datalisting}
\end{minipage}

\\
\end{tabular}


There are still some issues. If we were to try to translate \texttt{Graph 1} to a hypergraph, we would get the same hyper edge three times. That is to say, we would end up with a hypergraph containing only a single hyperedge.

This issue is resolved with the command line switch \texttt{-z}, which makes sure that no two vertices in the second class can have the same neighborhood.

\begin{code}
  $ ./genbg 2 3 -z -d0:1 | ./showg
\end{code}
This yields the output:

\begin{datalisting}
Graph 1, order 5.
  0 : 2 4;
  1 : 3 4;
  2 : 0;
  3 : 1;
  4 : 0 1;

\end{datalisting}
So in the end we have only a single hypergraph in this class. It has two vertices, each with it's own singleton hyperedge plus a hyperedge that contains both of the vertices.
Table \ref{tab:bipartite_hypergraph_correspondence} how this bipartite graph and it's corresponding hypergraph are related.

\begin{center}
\def\arraystretch{1.0}
\begin{table}
\begin{tabular}{c c}
  \def\svgwidth{0.45\columnwidth} \input{bipartite.pdf_tex}
  &
  \def\svgwidth{0.45\columnwidth} \input{hypergraph.pdf_tex}
\end{tabular}
\caption{The bipartite graph, and it's corresponding hypergraph.}
\label{tab:bipartite_hypergraph_correspondence}
\end{table}
\end{center}

Our hypothesis at this point is that a command such as \texttt{./genbg m n -z -d0:1} will generate all non-isomorphic hypergraphs of \texttt{m} vertices and \texttt{n} edges.

More precisely, we need to prove the following results.

\begin{definition}
The \emph{neighbourhood} $\mathcal{N}(v) \subset V(G)$ of a vertex $v \in V(G)$ is defined as the set of vertices which have edges connected directly to $v$.
\end{definition}

\begin{definition}
A graph $G$ is said to be \emph{bipartite}, if there are $V_1, V_2 \subset{V(G)}$ such that $V(G) = V_1 \cup V_2$, such that $V_1 \cap V_2 = \emptyset$, and such that $\mathcal{N}(V_1) \subset V_2$ and $\mathcal{N}(V_2) \subset V_1$.
We denote these subsets by $V_1(G)$ and $V_2(G)$, and call them the \emph{first and second classes} of $G$, respectively.
\end{definition}

\begin{lemma}
Let $\mathfrak{B}$ be the set of all bipartite graphs where all vertices in the second class have distinct, non-empty neighbourhoods.
Let $\mathfrak{H}$ be the set of all hypergraphs.
Then there is a bijection, up to isomorphism, between $\mathfrak{B}$ and $\mathfrak{H}$.
\end{lemma}
\begin{proof}
Let $G \in \mathfrak{B}$. Define $\varphi(G) = (V_1(G), \mathcal{N}(V_2(G))) \in \mathfrak{H}$.
(This is a member of $\mathfrak{H}$ by hypothesis -- each element of $\mathcal{N}(V_2(G))$ is non-empty.)
In this way we have constructed a mapping from $\mathfrak{B}$ to $\mathfrak{H}$, which we claim has the desired properties.

\textbf{Injectivity:}
To show \emph{injectivity up to isomorphism}, we should show $G \cong G' \Leftrightarrow \varphi(G) \cong \varphi(G')$, for any two $G,G' \in \mathfrak{B}$.
Suppose that there is a graph isomorphism $f: G \isomarrow G'$.
Define $h: V_1(G) \rightarrow V_1(G')$ by $h(v) = f(v)$. In other words, $h$ is just $f$ restricted to $V_1(G)$.
We must now show that $h$ yields a bijection between $\mathcal{N}(V_2(G))$ and $\mathcal{N}(V_2(G'))$.
Since $\mathcal{N}(V_2(G)) \subset V_1(G)$ we get $h = f$ on $\mathcal{N}(V_2(G))$.
Now, since $f$ is an isomorphism, $f$ indeed yields a bijection between $\mathcal{N}(V_2(G))$ and $\mathcal{N}(V_2(G'))$. Therefore, so does $h$.
We have shown $G \cong G' \Rightarrow \varphi(G) \cong \varphi(G')$.
To show the converse, suppose that we have a hypergraph isomorphism between $h: \varphi(G) \isomarrow \varphi(G')$.
We define $f: G \rightarrow G'$ in two cases: $f \|_{V_1(G)} = h$ and $f \|_{V_2(G)} = \mathcal{N}^{-1} h \mathcal{N}$.
We should check that the second expression is well defined. All vertices in $V_2(G)$ have distinct neighbourhoods, which is just another way of saying that $\mathcal{N}$ yields a bijection between vertices in $V_2(G)$ and their neighbourhoods, for all $G$. We also have that $h$ forms a bijection between the neighbourhoods of $V_2(G)$ and $V_2(G')$, by it's definition. These two facts are sufficcient to show that $f \|_{V_2(G)}$ is well defined.
To show that $f$ is an isomorphism, we should also check that whenever $v_1 \sim v_2$ in $G$, we get $f(v_1) \sim f(v_2)$ in $G'$, and vice-versa.
Suppose that $v_1 \sim v_2$ in $G$, where $v_1 \in V_1(G)$ and $v_2 \in V_2(G)$. It follows that $v_1 \in \mathcal{N}(v_2)$.
Therefore $h(v_1) \in h \mathcal{N}( v_2 )$.
Now, $h(v_1) = f(v_1)$ and $h \mathcal{N} (v_2) = \mathcal{N} f (v_2) $, so we have $f(v_1) \in \mathcal{N} f (v_2)$, which implies $f(v_1) \sim f(v_2)$.
This sequence of implications also holds in reverse, and so we get $f(v_1) \sim f(v_2) \Rightarrow v_1 \sim v_2$. (Note that $h(v_1) \in h \mathcal{N}(v_2) \Rightarrow v_1 \in \mathcal{N}(v_2)$ holds because $h$ is a bijection.)

\textbf{Surjectivity:}
To show \emph{surjectivity up to isomorphism}, we should show that, given any $H \in \mathfrak{H}$, there is a $G \in \mathfrak{B}$ such that $\varphi(G) \cong H$.
So let $H = (V, \mathcal{F}) \in \mathfrak{H}$ be any hypergraph.
Clearly, we can define a set $W$ such that $| W | = | \mathcal{F} |$ and $V \cap W = \emptyset$.
Since $| W | = | \mathcal{F} |$ is finite, there exists a bijection $\phi: W \rightarrow \mathcal{F}$.
We define $G \in \mathfrak{B}$ by $V_1(G) = V$, $V_2(G) = W$, and $v \sim w \Leftrightarrow v \in \phi(w)$, where $v \in V$ and $w \in W$.
By definition, we then have $\varphi(G) = (V, \mathcal{N}(W))$.
We can also see that $\mathcal{N} \|_{W} = \phi$, and so $\mathcal{N}(W) = \mathcal{F}$.
Thus, we get $\varphi(G) = (V, \mathcal{F}) = H$.
\qed
\end{proof}

We need a bit more precision, as provided by the following result, which follow immediately from the definition of $\varphi$.

We might want to restrict the bijection on the number of vertices and hyperedges.

\begin{corollary}
Let $\mathfrak{B}_{n,m} \subset \mathfrak{B}$ be those bipartite graphs in $\mathfrak{B}$ with $n$ vertices in the first class and $m$ vertices in the second class.
Let $\mathfrak{H}_{n,m}$ and the set of all non-isomorphic hypergraphs with $n$ vertices and $m$ hyperedges.
Then there is a bijection between $\mathfrak{B}_{n,m}$ and $\mathfrak{H}_{n,m}$.
\end{corollary}

We can also restrict on the size of the hyperedges and the number of hyperedges a given vertex can ocurr in.

\begin{corollary}
Let $d,d',D,D'$ be integers.
Let $\mathfrak{B}_{d,D,d',D'} \subset \mathfrak{B}$ be those bipartite graphs $G \in \mathfrak{B}$ which satisfy $d \leq | \mathcal{N}(V_1(G)) | \leq D$ and $d' \leq | \mathcal{N}(V_2(G) | \leq D'$.
Let $\mathfrak{H}_{d,D,d',D'} \subset \mathfrak{H}$ be those hypergraphs $(V, \mathcal{F}) \in \mathfrak{B}$ with vertices that are in at least $d$ and at most $D$ hyperedges (inclusive), and $d' \leq | \mathcal{F} | \leq D'$.
Then there is a bijection between $\mathfrak{B}_{d,D,d',D'}$ and $\mathfrak{H}_{d,D,d',D'}$.
\end{corollary}

Or, we can restrict it in both ways.

\begin{corollary}
\label{cor:hypergraph_bipartite_bijection}
There is a bijection between $\mathfrak{B}_{n,m} \cap \mathfrak{B}_{d,D,d',D'}$ and $\mathfrak{H}_{n,m} \cap \mathfrak{H}_{d,D,d',D'}$.
\end{corollary}


\section{Experiments}
\label{sec:experiments}
\subsection{Introduction}


We want to find out how ``good'' our MCTS strategy is for various numbers of iterations.
Of course we expect that it becomes better at higher numbers of iterations, but how many are necessary?
We would also want to know at what number, or even if, it becomes virtually perfect.


One way to find out when our MCTS algorithm becomes virtually perfect is to match it against a perfect opponent, i.e. a minimax strategy.
We can run perfect as First vs MCTS at $n$ iterations as Second many times, and see how many times MCTS will win, loose and tie.

We also want to classify the game being played as a first player win, second player win or neither player wins.
As discussed in section \ref{sec:minimax}, this means running perfect vs perfect for the game and recording the outcome.


\subsection{Classes of games to study}

The number of positional games grows rapidly with the number of vertices (or positions).
If we want to study interesting games of higher number of vertices, we are going to need to focus on particular classes.

In the experiment outlined below, we will focus on games which have hyper-edges (winning sets) with either two or three vertices.
In order to limit the number of vertices in a hyper-edge, we can use the arguments \texttt{-dm:n} and \texttt{-Dm:n} with \texttt{genbg}.
The argument \texttt{-dm:n} gives a lower bounds, $m$ and $n$, for the minimum degrees of the first and second classes of vertices, respectively.
Recall that the first class of vertices correspond to the vertices of the hyper-graph, and the second class refers to the hyper-edges, as described in \ref{sec:nautycommandline}.
Similarly, the argument \texttt{-Dm:n} gives upper bounds on the maximum degrees for the two classes.
Therefore, if we wanted to generate the class of hyper-graphs with 4 vertices and 2 hyper-edges which either have 2 or 3 vertices in them, we would run:

\begin{code}
  $ ./genbg -z 4 2 -d0:2 -D2:3 | ./showg

Graph 1, order 6.
  0 : 4 5;
  1 : 4;
  2 : 5;
  3 : ;
  4 : 0 1;
  5 : 0 2;

Graph 2, order 6.
  0 : 4;
  1 : 4;
  2 : 5;
  3 : 5;
  4 : 0 1;
  5 : 2 3;

Graph 3, order 6.
  0 : 4 5;
  1 : 4 5;
  2 : 5;
  3 : ;
  4 : 0 1;
  5 : 0 1 2;

Graph 4, order 6.
  0 : 4 5;
  1 : 4;
  2 : 5;
  3 : 5;
  4 : 0 1;
  5 : 0 2 3;

Graph 5, order 6.
  0 : 4 5;
  1 : 4 5;
  2 : 4;
  3 : 5;
  4 : 0 1 2;
  5 : 0 1 3;

\end{code}

\subsection{Experiment 1}
\label{sec:experiment1}

This chapter describes experiment 1. Mainly the setup and contents -- results are dealt with in the next chapter.
We are dealing with the set of all hypergraphs with hyperedges of size either two or three, from the following classes:

\begin{tabular}{ c | c }
\#vertices & \#hyperedges \\ \hline
2&1 \\ \hline
3&1-4 \\ \hline
4&1-10 \\ \hline
5&1-20 \\ \hline
6&1-13 \\ \hline
\end{tabular}

That ends up being about 4.3 million hypergraphs. 4320006, to be exact.

For each of those hypergraphs, we run three tournaments: minimax as first versus MCTS with 10, 20 and 30 iterations, as second.

Each tournament consists of 100 games. Within a given tournament, we record the number of first wins, second wins and the number of ties.
We also classify the games as first, second or neither player win, by playing two optimal players against each other and recording the result.
This all ends up in a big database, which is discussed more in detail in chapter \ref{sec:database_queries}.


\section{Appendix A: Database queries}
\label{sec:database_queries}
This section covers a number of interesting queries that one could make into the database from one of the experiments described above.

It is assumed that you have SQLite 3 installed.
To begin making queries into the database \texttt{mydatabase.db} (in the current working directory), you would issue the shell command:
\begin{code}
$ sqlite3 mydatabase.db
\end{code}
You will then be greeted with a prompt like
\begin{code}
sqlite> 
\end{code}
where you can begin typing the queries and commands covered below.

For the sake of completeness, we will run all the subsequent queries against the database \texttt{twothree.db}, which corresponds to our experiment, described in section \ref{sec:experiment1}.

\subsection{Database overview}

The following commands are not queries, but they are very important to know.
\begin{code}
sqlite> .tables
hypergraphs    mctsvsperfect  perfect 
\end{code}
These are the tables in \texttt{twothree.db}. The \texttt{hypergraphs} table contains the games we want to play, along with some meta information about the games. (See next command.)
The table \texttt{perfect} contains the outcome for a given game when perfect First plays against perfect Second. (The perfect strategies are implemented using minimax, as covered in section \ref{sec:minimax}.)
The table \texttt{mctsvsperfect} contains a number of sample outcomes when MCTS, or more precisely UCT, (section \ref{sec:uct}) plays as First against a perfect opponent as Second.

To get more precise information about what's contained in the above tables, issue the following command:

\begin{code}
sqlite> .schema
CREATE TABLE hypergraphs
(hypergraph STRING PRIMARY KEY NOT NULL,
 numvertices INTEGER NOT NULL,
 numedges INTEGER NOT NULL,
 representation STRING NOT NULL);
CREATE TABLE mctsvsperfect
(hypergraph STRING NOT NULL REFERENCES hypergraphs(hypergraph),
 numiterations INTEGER NOT NULL,
 numfirstwins INTEGER, numsecondwins INTEGER,
 numneitherwins INTEGER,
 UNIQUE(hypergraph, numiterations));
CREATE TABLE perfect
(hypergraph STRING PRIMARY KEY NOT NULL REFERENCES hypergraphs(hypergraph),
 winner STRING);
\end{code}
This command not only tells you the names, types and constraints of the collumns making up the table, but it does so by telling you the exact command that was issued to create the table.
The important information here is the names and types.
We can see that \texttt{hypergraphs} has a collumn named \texttt{hypergraph}, which stores the hypergraph as a non-null \footnote{\texttt{NULL} is used to denote 'nothing', and is not appropriate here, which is why it is explicitly disallowed.} string (in the graph6 format). 
The \texttt{hypergraphs} table also contains the number of vertices and edges as well as a more human-readable representation.
Even though the last three collums of \texttt{hypergraphs} can readily be derived from the first collumn, they are nice to have there for convenience when making queries, as will be seen below.


\subsection{Overall database structure}


This section presents various ways of querying the overall structure of the database.

\subsubsection{Query: Vertices and edges}

Suppose that we are interested in finding out, roughly, the structure of the table \texttt{mctsvsperfect} in our database.
We might first be interested in knowing which ``classes'' of hypergraphs are in the table, in the sense that two hypergraphs are in the same class iff they have the same number of vertices and hyperedges.

The following command will print out all such classes in the format \texttt{\#vertices | \#edges}.
\begin{code}
sqlite> SELECT DISTINCT numvertices,numedges
   ...> FROM mctsvsperfect NATURAL JOIN hypergraphs;
\end{code}
The output is supressed here, since it is not ordered.
If we want ordered results, we extend the previous query a little bit:
\begin{code}
sqlite> SELECT DISTINCT numvertices,numedges
   ...> FROM mctsvsperfect NATURAL JOIN hypergraphs
   ...> ORDER BY numvertices, numedges;
\end{code}

\begin{tabular}{| p{0.5in} | p{0.5in} | p{0.5in} |}
\begin{minipage}{0.5in}
\begin{output}
2|1
3|1
3|2
3|3
3|4
4|1
4|2
4|3
4|4
4|5
4|6
4|7
4|8
4|9
4|10
5|1
\end{output}
\end{minipage}
&
\begin{minipage}{0.5in}
\begin{output}
5|2
5|3
5|4
5|5
5|6
5|7
5|8
5|9
5|10
5|11
5|12
5|13
5|14
5|15
5|16
5|17
\end{output}
\end{minipage}
&
\begin{minipage}{0.5in}
\begin{output}
5|18
5|19
5|20
6|1
6|2
6|3
6|4
6|5
6|6
6|7
6|8
6|9
6|10
6|11
6|12
6|13
\end{output}
\end{minipage}
\\
\end{tabular}

Now we can see why the \texttt{hypergraphs} table exists, and contains redundant information: we simply need to do a \texttt{NATURAL JOIN} with it in order to get the number of vertices and edges for our hypergraphs in \texttt{mctsvsperfect}.

As can be seen above, some hypergraphs from the class \texttt{4 | 7} are present in \texttt{mctsvsperfect}. How many?
\begin{code}
sqlite> SELECT COUNT (*) FROM
   ...> ((SELECT DISTINCT hypergraph FROM mctsvsperfect) NATURAL JOIN hypergraphs)
   ...> WHERE numvertices = 4 AND numedges = 7;
11
\end{code}
Note that we add the qualifier \texttt{DISTINCT} when pulling hypergraphs from \texttt{mctsvsperfect}, since each hypergraph in this table ocurrs three times. (See the experiment structure in \ref{sec:experiment1}.)

If we want to know this number for all combinations of \texttt{numvertices} and \texttt{numedges} stored in \texttt{mctsvsperfect}, we can issue the following query
\begin{code}
sqlite> SELECT numvertices, numedges, count(hypergraph)
   ...> FROM ((SELECT DISTINCT hypergraph FROM mctsvsperfect) NATURAL JOIN hypergraphs)
   ...> GROUP BY numvertices, numedges;
\end{code}
\begin{tabular}{|c|c|c|}
\begin{minipage}{1.0in}
\begin{output}
2|1|1
3|1|2
3|2|2
3|3|2
3|4|1
4|1|2
4|2|5
4|3|11
4|4|17
4|5|18
4|6|17
4|7|11
4|8|5
4|9|2
4|10|1
5|1|2
\end{output}
\end{minipage}
&
\begin{minipage}{1.0in}
\begin{output}
5|2|7
5|3|24
5|4|75
5|5|192
5|6|431
5|7|806
5|8|1259
5|9|1644
5|10|1806
5|11|1644
5|12|1259
5|13|806
5|14|431
5|15|192
5|16|75
5|17|24
\end{output}
\end{minipage}
&
\begin{minipage}{1.0in}
\begin{output}
5|18|7
5|19|2
5|20|1
6|1|2
6|2|8
6|3|35
6|4|163
6|5|715
6|6|2958
6|7|11011
6|8|36277
6|9|105070
6|10|267522
6|11|600130
6|12|1190410
6|13|2094921
\end{output}
\end{minipage}
\\
\end{tabular}

The attentive reader will note the number of hypergraphs listed in the above table adds up to 4320006 -- the number of hypergraphs in the database:
\begin{code}
sqlite> SELECT COUNT (*) FROM hypergraphs;
4320006
\end{code}

\subsection{Results at large}

The following query shows the percentage of First wins, Second wins and Neither wins, respectively, for the games classified as a first player win.
\begin{code}
sqlite> SELECT SUM(numfirstwins), SUM(numsecondwins),
   ...>        SUM(numneitherwins), numiterations
   ...> FROM perfect NATURAL JOIN mctsvsperfect
   ...> WHERE winner = "First"
   ...> GROUP BY numiterations;
323672354|101627304|6511442|10
253345343|172140755|6325002|20
401797389|24320391|5693320|30
\end{code}
Or, approximately and in normalized form:
\begin{code}
0.75 | 0.24 | 0.02 | 10
0.59 | 0.40 | 0.01 | 20
0.93 | 0.06 | 0.01 | 30
\end{code}

Here is the same query, but investigating games where neither player can win if both play perfectly:

\begin{code}
sqlite> SELECT SUM(numfirstwins), SUM(numsecondwins),
   ...> SUM(numneitherwins), numiterations
   ...> FROM perfect NATURAL JOIN mctsvsperfect
   ...> WHERE winner = "Neither"
   ...> GROUP BY numiterations;
0|29642|159858|10
0|19103|170397|20
0|12800|176700|30
\end{code}

Again, in normalized form:

\begin{code}
0.00 | 0.15 | 0.84 | 10
0.00 | 0.10 | 0.90 | 20
0.00 | 0.07 | 0.93 | 30
\end{code}

\subsection{Results in perfect play}

Here is a query which will summarize the results from our games when two perfect players play against each other:

\begin{output}
sqlite> SELECT hc.numvertices, hc.numedges,
   ...> IFNULL(numfirstwins, 0), IFNULL(numsecondwins, 0), IFNULL(numneitherwins, 0)
   ...> FROM (SELECT numvertices, numedges FROM hypergraphs GROUP BY numvertices, numedges) hc
   ...> LEFT OUTER JOIN
   ...>   (SELECT numvertices, numedges, count(*) AS numfirstwins
   ...>    FROM hypergraphs NATURAL JOIN perfect
   ...>    WHERE winner = 'First' GROUP BY numvertices, numedges) fwc
   ...> ON (hc.numvertices = fwc.numvertices AND hc.numedges = fwc.numedges)
   ...> LEFT OUTER JOIN
   ...>   (SELECT numvertices, numedges, count(*) AS numsecondwins
   ...>    FROM hypergraphs NATURAL JOIN perfect
   ...>    WHERE winner = 'Second' GROUP BY numvertices, numedges) swc
   ...> ON (hc.numvertices = swc.numvertices AND hc.numedges = swc.numedges)
   ...> LEFT OUTER JOIN
   ...>   (SELECT numvertices, numedges, count(*) AS numneitherwins
   ...>    FROM hypergraphs NATURAL JOIN perfect
   ...>    WHERE winner = 'Neither' GROUP BY numvertices, numedges) nwc
   ...> ON (hc.numvertices = nwc.numvertices AND hc.numedges = nwc.numedges);
\end{output}
The output is as follows:

\begin{tabular}{c | c | c}
\begin{minipage}{1.0in}
\begin{output}
2|1|0|0|1|
3|1|0|0|2|
3|2|1|0|1|
3|3|2|0|0|
3|4|1|0|0|
4|1|0|0|2|
4|2|1|0|4|
4|3|6|0|5|
4|4|12|0|5|
4|5|16|0|2|
4|6|16|0|1|
4|7|11|0|0|
4|8|5|0|0|
4|9|2|0|0|
4|10|1|0|0|
5|1|0|0|2|
\end{output}
\end{minipage}
&
\begin{minipage}{1.0in}
\begin{output}
5|2|1|0|6|
5|3|11|0|13|
5|4|53|0|22|
5|5|171|0|21|
5|6|418|0|13|
5|7|802|0|4|
5|8|1258|0|1|
5|9|1644|0|0|
5|10|1806|0|0|
5|11|1644|0|0|
5|12|1259|0|0|
5|13|806|0|0|
5|14|431|0|0|
5|15|192|0|0|
5|16|75|0|0|
5|17|24|0|0|
\end{output}
\end{minipage}
&
\begin{minipage}{1.0in}
\begin{output}
5|18|7|0|0|
5|19|2|0|0|
5|20|1|0|0|
6|1|0|0|2|
6|2|1|0|7|
6|3|12|0|23|
6|4|94|0|69|
6|5|559|0|156|
6|6|2684|0|274|
6|7|10660|0|351|
6|8|35931|0|346|
6|9|104799|0|271|
6|10|267357|0|165|
6|11|600047|0|83|
6|12|1190378|0|32|
6|13|2094910|0|11|
\end{output}
\end{minipage}
\\
\end{tabular}

The first two collumns contain the number of vertices and edges, respectively. The following three collumns contains the number of times First, Second and Neither won in that category.

\subsection{More detailed results}

The following tables show percentages of First, Second and Neither wins, respectively, for a given number of 2-edges and 3-edges.
(Recall that all edges are of size either 2 or 3, as described in section \ref{sec:experiment1}.)
Note that the number of 3-edges increase along rows, and the number of 2-edges increase along collumns.

The query is split up into First player win games (games where First wins if both players play perfectly), and Neither player win games (games where neither player wins if both play perfectly).
Table \ref{tab:detailed_results_first} shows the query for First player win games. Table \ref{tab:detailed_results_neither} shows the query for Neither player win games.

\begin{landscape}
\bgroup
\setlength{\tabcolsep}{.16em}
\def\arraystretch{0.5}
\begin{table}
\begin{tabular}{|>{\small\ttfamily}r||>{\small\ttfamily}c|>{\small\ttfamily}c|>{\small\ttfamily}c|>{\small\ttfamily}c|>{\small\ttfamily}c|>{\small\ttfamily}c|>{\small\ttfamily}c|>{\small\ttfamily}c|>{\small\ttfamily}c|>{\small\ttfamily}c|>{\small\ttfamily}c|>{\small\ttfamily}c|>{\small\ttfamily}c|>{\small\ttfamily}c|}
\hline
\begin{tabular}{>{\tiny\ttfamily}c}
\#3-edges\\
\hline
\#2-edges\\
\end{tabular}
&0&1&2&3&4&5&6&7&8&9&10&11&12&13
\\ \hline
\begin{tabular}{>{\small\ttfamily}c|>{\tiny\ttfamily}c}
\multirow{3}{*}{0} & 10 \\
& 20 \\
& 30 \\
\end{tabular}
&&&&&
\begin{tabular}{>{\tiny\ttfamily}c}3,0,97\\4,0,97\\1,0,99\\\end{tabular}
&
\begin{tabular}{>{\tiny\ttfamily}c}9,5,86\\2,5,93\\2,6,92\\\end{tabular}
&
\begin{tabular}{>{\tiny\ttfamily}c}9,13,78\\4,10,86\\3,10,88\\\end{tabular}
&
\begin{tabular}{>{\tiny\ttfamily}c}9,21,69\\4,14,82\\4,12,85\\\end{tabular}
&
\begin{tabular}{>{\tiny\ttfamily}c}12,25,62\\7,17,77\\5,14,81\\\end{tabular}
&
\begin{tabular}{>{\tiny\ttfamily}c}18,28,54\\11,18,71\\8,16,76\\\end{tabular}
&
\begin{tabular}{>{\tiny\ttfamily}c}26,29,46\\19,19,63\\14,18,68\\\end{tabular}
&
\begin{tabular}{>{\tiny\ttfamily}c}34,27,39\\30,17,53\\23,18,59\\\end{tabular}
&
\begin{tabular}{>{\tiny\ttfamily}c}47,23,31\\43,14,43\\32,17,50\\\end{tabular}
&
\begin{tabular}{>{\tiny\ttfamily}c}56,19,24\\53,15,32\\39,22,39\\\end{tabular}
\\ \hline
\begin{tabular}{>{\small\ttfamily}c|>{\tiny\ttfamily}c}
\multirow{3}{*}{1} & 10 \\
& 20 \\
& 30 \\
\end{tabular}
&&&
\begin{tabular}{>{\tiny\ttfamily}c}22,11,68\\5,7,89\\8,20,73\\\end{tabular}
&
\begin{tabular}{>{\tiny\ttfamily}c}26,19,55\\10,23,67\\9,31,60\\\end{tabular}
&
\begin{tabular}{>{\tiny\ttfamily}c}25,29,46\\12,29,59\\12,31,57\\\end{tabular}
&
\begin{tabular}{>{\tiny\ttfamily}c}20,38,42\\14,34,52\\14,33,53\\\end{tabular}
&
\begin{tabular}{>{\tiny\ttfamily}c}19,45,37\\16,37,47\\18,34,48\\\end{tabular}
&
\begin{tabular}{>{\tiny\ttfamily}c}19,49,32\\18,40,42\\22,36,43\\\end{tabular}
&
\begin{tabular}{>{\tiny\ttfamily}c}20,53,27\\21,43,36\\27,36,36\\\end{tabular}
&
\begin{tabular}{>{\tiny\ttfamily}c}23,55,22\\25,45,31\\34,36,30\\\end{tabular}
&
\begin{tabular}{>{\tiny\ttfamily}c}28,54,18\\28,46,26\\42,34,24\\\end{tabular}
&
\begin{tabular}{>{\tiny\ttfamily}c}35,50,15\\33,46,21\\51,31,19\\\end{tabular}
&
\begin{tabular}{>{\tiny\ttfamily}c}44,44,12\\39,44,17\\60,26,14\\\end{tabular}
&
\\ \hline
\begin{tabular}{>{\small\ttfamily}c|>{\tiny\ttfamily}c}
\multirow{3}{*}{2} & 10 \\
& 20 \\
& 30 \\
\end{tabular}
&
\begin{tabular}{>{\tiny\ttfamily}c}25,0,76\\51,0,49\\0,0,100\\\end{tabular}
&
\begin{tabular}{>{\tiny\ttfamily}c}32,15,54\\24,21,55\\7,26,67\\\end{tabular}
&
\begin{tabular}{>{\tiny\ttfamily}c}40,27,33\\22,36,42\\19,44,38\\\end{tabular}
&
\begin{tabular}{>{\tiny\ttfamily}c}41,34,26\\25,44,32\\24,45,31\\\end{tabular}
&
\begin{tabular}{>{\tiny\ttfamily}c}41,38,21\\27,48,25\\29,43,28\\\end{tabular}
&
\begin{tabular}{>{\tiny\ttfamily}c}42,41,18\\28,51,21\\34,41,25\\\end{tabular}
&
\begin{tabular}{>{\tiny\ttfamily}c}45,40,15\\28,54,17\\40,39,22\\\end{tabular}
&
\begin{tabular}{>{\tiny\ttfamily}c}49,39,12\\28,58,14\\46,36,18\\\end{tabular}
&
\begin{tabular}{>{\tiny\ttfamily}c}55,36,9\\27,61,11\\52,33,14\\\end{tabular}
&
\begin{tabular}{>{\tiny\ttfamily}c}60,33,7\\26,65,9\\59,30,11\\\end{tabular}
&
\begin{tabular}{>{\tiny\ttfamily}c}65,29,6\\24,69,7\\66,26,9\\\end{tabular}
&
\begin{tabular}{>{\tiny\ttfamily}c}71,25,4\\21,74,5\\72,22,6\\\end{tabular}
&&
\\ \hline
\begin{tabular}{>{\small\ttfamily}c|>{\tiny\ttfamily}c}
\multirow{3}{*}{3} & 10 \\
& 20 \\
& 30 \\
\end{tabular}
&
\begin{tabular}{>{\tiny\ttfamily}c}52,26,22\\71,0,30\\43,27,30\\\end{tabular}
&
\begin{tabular}{>{\tiny\ttfamily}c}54,29,18\\56,21,23\\43,35,22\\\end{tabular}
&
\begin{tabular}{>{\tiny\ttfamily}c}49,35,16\\47,34,19\\48,33,20\\\end{tabular}
&
\begin{tabular}{>{\tiny\ttfamily}c}43,43,14\\43,42,16\\54,31,16\\\end{tabular}
&
\begin{tabular}{>{\tiny\ttfamily}c}39,48,14\\40,46,14\\58,28,14\\\end{tabular}
&
\begin{tabular}{>{\tiny\ttfamily}c}38,50,12\\39,49,12\\63,26,11\\\end{tabular}
&
\begin{tabular}{>{\tiny\ttfamily}c}40,49,11\\39,51,10\\67,24,9\\\end{tabular}
&
\begin{tabular}{>{\tiny\ttfamily}c}43,48,9\\40,52,8\\72,21,7\\\end{tabular}
&
\begin{tabular}{>{\tiny\ttfamily}c}47,45,7\\41,52,7\\76,18,5\\\end{tabular}
&
\begin{tabular}{>{\tiny\ttfamily}c}52,42,6\\42,53,5\\80,16,4\\\end{tabular}
&
\begin{tabular}{>{\tiny\ttfamily}c}57,39,5\\43,53,4\\84,13,3\\\end{tabular}
&&&
\\ \hline
\begin{tabular}{>{\small\ttfamily}c|>{\tiny\ttfamily}c}
\multirow{3}{*}{4} & 10 \\
& 20 \\
& 30 \\
\end{tabular}
&
\begin{tabular}{>{\tiny\ttfamily}c}72,15,14\\73,12,15\\69,13,18\\\end{tabular}
&
\begin{tabular}{>{\tiny\ttfamily}c}73,23,3\\70,25,5\\77,16,7\\\end{tabular}
&
\begin{tabular}{>{\tiny\ttfamily}c}66,31,3\\60,36,4\\77,18,6\\\end{tabular}
&
\begin{tabular}{>{\tiny\ttfamily}c}61,37,2\\52,44,4\\79,17,4\\\end{tabular}
&
\begin{tabular}{>{\tiny\ttfamily}c}59,39,2\\47,49,4\\81,16,4\\\end{tabular}
&
\begin{tabular}{>{\tiny\ttfamily}c}58,40,2\\45,52,3\\83,14,3\\\end{tabular}
&
\begin{tabular}{>{\tiny\ttfamily}c}59,40,1\\45,53,3\\85,12,2\\\end{tabular}
&
\begin{tabular}{>{\tiny\ttfamily}c}59,39,1\\45,53,2\\88,11,2\\\end{tabular}
&
\begin{tabular}{>{\tiny\ttfamily}c}60,39,1\\46,52,2\\90,9,1\\\end{tabular}
&
\begin{tabular}{>{\tiny\ttfamily}c}61,38,1\\47,52,1\\92,7,1\\\end{tabular}
&
\begin{tabular}{>{\tiny\ttfamily}c}100,0,0\\100,0,0\\100,0,0\\\end{tabular}
&&&
\\ \hline
\begin{tabular}{>{\small\ttfamily}c|>{\tiny\ttfamily}c}
\multirow{3}{*}{5} & 10 \\
& 20 \\
& 30 \\
\end{tabular}
&
\begin{tabular}{>{\tiny\ttfamily}c}81,19,1\\73,26,1\\93,6,1\\\end{tabular}
&
\begin{tabular}{>{\tiny\ttfamily}c}76,22,2\\72,27,1\\93,7,0\\\end{tabular}
&
\begin{tabular}{>{\tiny\ttfamily}c}73,25,2\\65,35,1\\93,7,0\\\end{tabular}
&
\begin{tabular}{>{\tiny\ttfamily}c}73,25,2\\59,40,1\\93,6,0\\\end{tabular}
&
\begin{tabular}{>{\tiny\ttfamily}c}73,25,2\\56,44,1\\94,6,0\\\end{tabular}
&
\begin{tabular}{>{\tiny\ttfamily}c}74,25,1\\54,46,1\\95,5,0\\\end{tabular}
&
\begin{tabular}{>{\tiny\ttfamily}c}75,24,1\\52,47,0\\95,4,0\\\end{tabular}
&
\begin{tabular}{>{\tiny\ttfamily}c}75,24,1\\52,48,0\\96,4,0\\\end{tabular}
&
\begin{tabular}{>{\tiny\ttfamily}c}76,23,1\\51,49,0\\97,3,0\\\end{tabular}
&
\begin{tabular}{>{\tiny\ttfamily}c}100,0,0\\100,0,0\\99,1,0\\\end{tabular}
&
\begin{tabular}{>{\tiny\ttfamily}c}100,0,0\\100,0,0\\100,0,0\\\end{tabular}
&&&
\\ \hline
\begin{tabular}{>{\small\ttfamily}c|>{\tiny\ttfamily}c}
\multirow{3}{*}{6} & 10 \\
& 20 \\
& 30 \\
\end{tabular}
&
\begin{tabular}{>{\tiny\ttfamily}c}92,8,0\\70,30,0\\99,1,0\\\end{tabular}
&
\begin{tabular}{>{\tiny\ttfamily}c}86,14,1\\72,28,0\\99,1,0\\\end{tabular}
&
\begin{tabular}{>{\tiny\ttfamily}c}84,16,1\\69,31,0\\98,2,0\\\end{tabular}
&
\begin{tabular}{>{\tiny\ttfamily}c}83,17,1\\67,33,0\\99,1,0\\\end{tabular}
&
\begin{tabular}{>{\tiny\ttfamily}c}82,17,1\\66,34,0\\99,1,0\\\end{tabular}
&
\begin{tabular}{>{\tiny\ttfamily}c}82,17,0\\66,34,0\\99,1,0\\\end{tabular}
&
\begin{tabular}{>{\tiny\ttfamily}c}82,18,0\\66,34,0\\99,1,0\\\end{tabular}
&
\begin{tabular}{>{\tiny\ttfamily}c}82,18,0\\66,34,0\\99,1,0\\\end{tabular}
&
\begin{tabular}{>{\tiny\ttfamily}c}100,0,0\\100,0,0\\100,0,0\\\end{tabular}
&
\begin{tabular}{>{\tiny\ttfamily}c}100,0,0\\100,0,0\\100,0,0\\\end{tabular}
&
\begin{tabular}{>{\tiny\ttfamily}c}100,0,0\\100,0,0\\100,0,0\\\end{tabular}
&&&
\\ \hline
\begin{tabular}{>{\small\ttfamily}c|>{\tiny\ttfamily}c}
\multirow{3}{*}{7} & 10 \\
& 20 \\
& 30 \\
\end{tabular}
&
\begin{tabular}{>{\tiny\ttfamily}c}95,5,0\\78,22,0\\99,1,0\\\end{tabular}
&
\begin{tabular}{>{\tiny\ttfamily}c}95,5,0\\86,14,0\\100,0,0\\\end{tabular}
&
\begin{tabular}{>{\tiny\ttfamily}c}96,4,0\\86,14,0\\100,0,0\\\end{tabular}
&
\begin{tabular}{>{\tiny\ttfamily}c}96,4,0\\86,14,0\\100,0,0\\\end{tabular}
&
\begin{tabular}{>{\tiny\ttfamily}c}96,4,0\\86,14,0\\100,0,0\\\end{tabular}
&
\begin{tabular}{>{\tiny\ttfamily}c}96,4,0\\86,14,0\\100,0,0\\\end{tabular}
&
\begin{tabular}{>{\tiny\ttfamily}c}96,4,0\\87,13,0\\100,0,0\\\end{tabular}
&
\begin{tabular}{>{\tiny\ttfamily}c}100,0,0\\100,0,0\\100,0,0\\\end{tabular}
&
\begin{tabular}{>{\tiny\ttfamily}c}100,0,0\\100,0,0\\100,0,0\\\end{tabular}
&
\begin{tabular}{>{\tiny\ttfamily}c}100,0,0\\100,0,0\\100,0,0\\\end{tabular}
&
\begin{tabular}{>{\tiny\ttfamily}c}100,0,0\\100,0,0\\100,0,0\\\end{tabular}
&&&
\\ \hline
\begin{tabular}{>{\small\ttfamily}c|>{\tiny\ttfamily}c}
\multirow{3}{*}{8} & 10 \\
& 20 \\
& 30 \\
\end{tabular}
&
\begin{tabular}{>{\tiny\ttfamily}c}91,9,0\\96,4,0\\100,0,0\\\end{tabular}
&
\begin{tabular}{>{\tiny\ttfamily}c}98,2,0\\89,11,0\\100,0,0\\\end{tabular}
&
\begin{tabular}{>{\tiny\ttfamily}c}99,1,0\\88,12,0\\100,0,0\\\end{tabular}
&
\begin{tabular}{>{\tiny\ttfamily}c}99,1,0\\87,13,0\\100,0,0\\\end{tabular}
&
\begin{tabular}{>{\tiny\ttfamily}c}99,1,0\\86,14,0\\100,0,0\\\end{tabular}
&
\begin{tabular}{>{\tiny\ttfamily}c}99,1,0\\86,14,0\\100,0,0\\\end{tabular}
&
\begin{tabular}{>{\tiny\ttfamily}c}100,0,0\\100,0,0\\100,0,0\\\end{tabular}
&
\begin{tabular}{>{\tiny\ttfamily}c}100,0,0\\100,0,0\\100,0,0\\\end{tabular}
&
\begin{tabular}{>{\tiny\ttfamily}c}100,0,0\\100,0,0\\100,0,0\\\end{tabular}
&
\begin{tabular}{>{\tiny\ttfamily}c}100,0,0\\100,0,0\\100,0,0\\\end{tabular}
&
\begin{tabular}{>{\tiny\ttfamily}c}100,0,0\\100,0,0\\100,0,0\\\end{tabular}
&&&
\\ \hline
\begin{tabular}{>{\small\ttfamily}c|>{\tiny\ttfamily}c}
\multirow{3}{*}{9} & 10 \\
& 20 \\
& 30 \\
\end{tabular}
&
\begin{tabular}{>{\tiny\ttfamily}c}98,2,0\\100,0,0\\100,0,0\\\end{tabular}
&
\begin{tabular}{>{\tiny\ttfamily}c}98,2,0\\100,0,0\\100,0,0\\\end{tabular}
&
\begin{tabular}{>{\tiny\ttfamily}c}99,1,0\\100,0,0\\100,0,0\\\end{tabular}
&
\begin{tabular}{>{\tiny\ttfamily}c}99,1,0\\100,0,0\\100,0,0\\\end{tabular}
&
\begin{tabular}{>{\tiny\ttfamily}c}99,1,0\\100,0,0\\100,0,0\\\end{tabular}
&
\begin{tabular}{>{\tiny\ttfamily}c}100,0,0\\100,0,0\\100,0,0\\\end{tabular}
&
\begin{tabular}{>{\tiny\ttfamily}c}100,0,0\\100,0,0\\100,0,0\\\end{tabular}
&
\begin{tabular}{>{\tiny\ttfamily}c}100,0,0\\100,0,0\\100,0,0\\\end{tabular}
&
\begin{tabular}{>{\tiny\ttfamily}c}100,0,0\\100,0,0\\100,0,0\\\end{tabular}
&
\begin{tabular}{>{\tiny\ttfamily}c}100,0,0\\100,0,0\\100,0,0\\\end{tabular}
&
\begin{tabular}{>{\tiny\ttfamily}c}100,0,0\\100,0,0\\100,0,0\\\end{tabular}
&&&
\\ \hline
\begin{tabular}{>{\small\ttfamily}c|>{\tiny\ttfamily}c}
\multirow{3}{*}{10} & 10 \\
& 20 \\
& 30 \\
\end{tabular}
&
\begin{tabular}{>{\tiny\ttfamily}c}100,0,0\\100,0,0\\100,0,0\\\end{tabular}
&
\begin{tabular}{>{\tiny\ttfamily}c}100,0,0\\100,0,0\\100,0,0\\\end{tabular}
&
\begin{tabular}{>{\tiny\ttfamily}c}100,0,0\\100,0,0\\100,0,0\\\end{tabular}
&
\begin{tabular}{>{\tiny\ttfamily}c}100,0,0\\100,0,0\\100,0,0\\\end{tabular}
&
\begin{tabular}{>{\tiny\ttfamily}c}100,0,0\\100,0,0\\100,0,0\\\end{tabular}
&
\begin{tabular}{>{\tiny\ttfamily}c}100,0,0\\100,0,0\\100,0,0\\\end{tabular}
&
\begin{tabular}{>{\tiny\ttfamily}c}100,0,0\\100,0,0\\100,0,0\\\end{tabular}
&
\begin{tabular}{>{\tiny\ttfamily}c}100,0,0\\100,0,0\\100,0,0\\\end{tabular}
&
\begin{tabular}{>{\tiny\ttfamily}c}100,0,0\\100,0,0\\100,0,0\\\end{tabular}
&
\begin{tabular}{>{\tiny\ttfamily}c}100,0,0\\100,0,0\\100,0,0\\\end{tabular}
&
\begin{tabular}{>{\tiny\ttfamily}c}100,0,0\\100,0,0\\100,0,0\\\end{tabular}
&&&
\\ \hline
\begin{tabular}{>{\small\ttfamily}c|>{\tiny\ttfamily}c}
\multirow{3}{*}{11} & 10 \\
& 20 \\
& 30 \\
\end{tabular}
&
\begin{tabular}{>{\tiny\ttfamily}c}100,0,0\\100,0,0\\100,0,0\\\end{tabular}
&
\begin{tabular}{>{\tiny\ttfamily}c}100,0,0\\100,0,0\\100,0,0\\\end{tabular}
&
\begin{tabular}{>{\tiny\ttfamily}c}100,0,0\\100,0,0\\100,0,0\\\end{tabular}
&&&&&&&&&&&
\\ \hline
\begin{tabular}{>{\small\ttfamily}c|>{\tiny\ttfamily}c}
\multirow{3}{*}{12} & 10 \\
& 20 \\
& 30 \\
\end{tabular}
&
\begin{tabular}{>{\tiny\ttfamily}c}100,0,0\\100,0,0\\100,0,0\\\end{tabular}
&
\begin{tabular}{>{\tiny\ttfamily}c}100,0,0\\100,0,0\\100,0,0\\\end{tabular}
&&&&&&&&&&&&
\\ \hline
\begin{tabular}{>{\small\ttfamily}c|>{\tiny\ttfamily}c}
\multirow{3}{*}{13} & 10 \\
& 20 \\
& 30 \\
\end{tabular}
&
\begin{tabular}{>{\tiny\ttfamily}c}100,0,0\\100,0,0\\100,0,0\\\end{tabular}
&&&&&&&&&&&&&
\\ \hline
\end{tabular}
\caption{Results for First player win games. 10, 20, 30 iterations.}
\label{tab:detailed_results_first}
\end{table}
\egroup
\end{landscape}

\begin{landscape}
\bgroup
\setlength{\tabcolsep}{.16em}
\def\arraystretch{0.5}%  1 is the default, change whatever you need
\begin{table}
\begin{tabular}{|>{\small\ttfamily}r||>{\small\ttfamily}c|>{\small\ttfamily}c|>{\small\ttfamily}c|>{\small\ttfamily}c|>{\small\ttfamily}c|>{\small\ttfamily}c|>{\small\ttfamily}c|>{\small\ttfamily}c|>{\small\ttfamily}c|>{\small\ttfamily}c|>{\small\ttfamily}c|>{\small\ttfamily}c|>{\small\ttfamily}c|}
\hline
\begin{tabular}{>{\tiny\ttfamily}c}
\#3-edges \\
\hline
\#2-edges \\
\end{tabular}
&0&1&2&3&4&5&6&7&8&9&10&11&12
\\ \hline
\begin{tabular}{>{\small\ttfamily}c|>{\tiny\ttfamily}c}
\multirow{3}{*}{0} & 10 \\
& 20 \\
& 30 \\
\end{tabular}
&&
\begin{tabular}{>{\tiny\ttfamily}c}
0,0,100\\
0,0,100\\
0,0,100\\
\end{tabular}
&
\begin{tabular}{>{\tiny\ttfamily}c}
0,0,100\\
0,0,100\\
0,0,100\\
\end{tabular}
&
\begin{tabular}{>{\tiny\ttfamily}c}
0,0,100\\
0,0,100\\
0,0,100\\
\end{tabular}
&
\begin{tabular}{>{\tiny\ttfamily}c}
0,1,99\\
0,1,99\\
0,1,99\\
\end{tabular}
&
\begin{tabular}{>{\tiny\ttfamily}c}
0,4,96\\
0,1,99\\
0,1,99\\
\end{tabular}
&
\begin{tabular}{>{\tiny\ttfamily}c}
0,9,91\\
0,3,97\\
0,2,98\\
\end{tabular}
&
\begin{tabular}{>{\tiny\ttfamily}c}
0,12,88\\
0,4,96\\
0,3,97\\
\end{tabular}
&
\begin{tabular}{>{\tiny\ttfamily}c}
0,12,88\\
0,4,96\\
0,3,97\\
\end{tabular}
&
\begin{tabular}{>{\tiny\ttfamily}c}
0,9,91\\
0,2,98\\
0,2,98\\
\end{tabular}
&
\begin{tabular}{>{\tiny\ttfamily}c}
0,4,96\\
0,1,100\\
0,0,100\\
\end{tabular}
&
\begin{tabular}{>{\tiny\ttfamily}c}
0,0,100\\
0,0,100\\
0,0,100\\
\end{tabular}
&
\begin{tabular}{>{\tiny\ttfamily}c}
0,0,100\\
0,0,100\\
0,0,100\\
\end{tabular}
\\ \hline
\begin{tabular}{>{\small\ttfamily}c|>{\tiny\ttfamily}c}
\multirow{3}{*}{1} & 10 \\
& 20 \\
& 30 \\
\end{tabular}
&
\begin{tabular}{>{\tiny\ttfamily}c}
0,0,100\\
0,0,100\\
0,0,100\\
\end{tabular}
&
\begin{tabular}{>{\tiny\ttfamily}c}
0,1,99\\
0,0,100\\
0,2,98\\
\end{tabular}
&
\begin{tabular}{>{\tiny\ttfamily}c}
0,5,95\\
0,5,95\\
0,8,92\\
\end{tabular}
&
\begin{tabular}{>{\tiny\ttfamily}c}
0,9,91\\
0,9,91\\
0,11,89\\
\end{tabular}
&
\begin{tabular}{>{\tiny\ttfamily}c}
0,12,88\\
0,11,89\\
0,11,89\\
\end{tabular}
&
\begin{tabular}{>{\tiny\ttfamily}c}
0,15,85\\
0,13,87\\
0,11,89\\
\end{tabular}
&
\begin{tabular}{>{\tiny\ttfamily}c}
0,18,82\\
0,15,85\\
0,11,89\\
\end{tabular}
&
\begin{tabular}{>{\tiny\ttfamily}c}
0,22,78\\
0,17,83\\
0,13,87\\
\end{tabular}
&
\begin{tabular}{>{\tiny\ttfamily}c}
0,26,74\\
0,19,81\\
0,15,85\\
\end{tabular}
&
\begin{tabular}{>{\tiny\ttfamily}c}
0,30,70\\
0,20,80\\
0,17,83\\
\end{tabular}
&
\begin{tabular}{>{\tiny\ttfamily}c}
0,35,65\\
0,24,76\\
0,21,79\\
\end{tabular}
&
\begin{tabular}{>{\tiny\ttfamily}c}
0,42,58\\
0,24,76\\
0,26,74\\
\end{tabular}
&
\begin{tabular}{>{\tiny\ttfamily}c}
0,55,46\\
0,29,71\\
0,34,67\\
\end{tabular}
\\ \hline
\begin{tabular}{>{\small\ttfamily}c|>{\tiny\ttfamily}c}
\multirow{3}{*}{2} & 10 \\
& 20 \\
& 30 \\
\end{tabular}
&
\begin{tabular}{>{\tiny\ttfamily}c}
0,0,100\\
0,0,100\\
0,0,100\\
\end{tabular}
&
\begin{tabular}{>{\tiny\ttfamily}c}
0,3,97\\
0,2,98\\
0,2,98\\
\end{tabular}
&
\begin{tabular}{>{\tiny\ttfamily}c}
0,9,91\\
0,4,96\\
0,2,98\\
\end{tabular}
&
\begin{tabular}{>{\tiny\ttfamily}c}
0,15,85\\
0,7,93\\
0,2,98\\
\end{tabular}
&
\begin{tabular}{>{\tiny\ttfamily}c}
0,18,82\\
0,8,92\\
0,2,98\\
\end{tabular}
&
\begin{tabular}{>{\tiny\ttfamily}c}
0,19,81\\
0,9,91\\
0,2,98\\
\end{tabular}
&
\begin{tabular}{>{\tiny\ttfamily}c}
0,20,80\\
0,9,91\\
0,2,98\\
\end{tabular}
&
\begin{tabular}{>{\tiny\ttfamily}c}
0,20,80\\
0,10,90\\
0,2,98\\
\end{tabular}
&
\begin{tabular}{>{\tiny\ttfamily}c}
0,21,79\\
0,10,90\\
0,2,98\\
\end{tabular}
&
\begin{tabular}{>{\tiny\ttfamily}c}
0,19,81\\
0,10,90\\
0,2,98\\
\end{tabular}
&
\begin{tabular}{>{\tiny\ttfamily}c}
0,21,79\\
0,11,89\\
0,1,99\\
\end{tabular}
&
\begin{tabular}{>{\tiny\ttfamily}c}
0,18,82\\
0,10,90\\
0,2,98\\
\end{tabular}
&
\\ \hline
\begin{tabular}{>{\small\ttfamily}c|>{\tiny\ttfamily}c}
\multirow{3}{*}{3} & 10 \\
& 20 \\
& 30 \\
\end{tabular}
&
\begin{tabular}{>{\tiny\ttfamily}c}
0,0,100\\
0,0,100\\
0,0,100\\
\end{tabular}
&
\begin{tabular}{>{\tiny\ttfamily}c}
0,0,100\\
0,0,100\\
0,0,100\\
\end{tabular}
&
\begin{tabular}{>{\tiny\ttfamily}c}
0,0,100\\
0,0,100\\
0,0,100\\
\end{tabular}
&
\begin{tabular}{>{\tiny\ttfamily}c}
0,0,100\\
0,0,100\\
0,0,100\\
\end{tabular}
&
\begin{tabular}{>{\tiny\ttfamily}c}
0,0,100\\
0,0,100\\
0,0,100\\
\end{tabular}
&
\begin{tabular}{>{\tiny\ttfamily}c}
0,0,100\\
0,0,100\\
0,0,100\\
\end{tabular}
&
\begin{tabular}{>{\tiny\ttfamily}c}
0,0,100\\
0,0,100\\
0,0,100\\
\end{tabular}
&
\begin{tabular}{>{\tiny\ttfamily}c}
0,0,100\\
0,0,100\\
0,0,100\\
\end{tabular}
&
\begin{tabular}{>{\tiny\ttfamily}c}
0,0,100\\
0,0,100\\
0,0,100\\
\end{tabular}
&
\begin{tabular}{>{\tiny\ttfamily}c}
0,0,100\\
0,0,100\\
0,0,100\\
\end{tabular}
&
\begin{tabular}{>{\tiny\ttfamily}c}
0,0,100\\
0,0,100\\
0,0,100\\
\end{tabular}
&&
\\ \hline
\end{tabular}
\caption{Results for Neither player win games. 10, 20, 30 iterations.}
\label{tab:detailed_results_neither}
\end{table}
\egroup
\end{landscape}


\section{Appendix B: Code}
\label{sec:code}
\subsection {Overview}

In this section, we give an overview of the overall structure of the code and the \texttt{PoGa} library.


Some parts of the \texttt{PoGa} library, such as the implementations of the games Tic-Tac-Toe and Hex, are not relevant and are thus not brought up here.
The same goes for the entire \texttt{Graphics} submodule, which implements support for a human player for the cases of Tic-Tac-Toe and Hex.
(The reason that we cannot just implement support for a human player for all positional games is that we cannot implement attractive user interfaces for the class of all games.)


\subsection {Positional game library}

In this section, we introduce the \texttt{PoGa} library.
One part of \texttt{PoGa} is mainly an interface used for representing and playing positional games.
However, \texttt{PoGa} also provides the two strategies discussed earlier: minimax with alpha-beta pruning, and the UCT variant of MCTS.

\texttt{PoGa} also contains implementations of arbitrary sized ($m \times n$) Tic-Tac-Toe and Hex.

\subsubsection {Game.hs}

This module defines the \texttt{Position} type-class, which is the central interface in the \texttt{PoGa} module.
It also defines the notion of a \texttt{Strategy} and how to play two strategies against each other in \texttt{playGame}.

Note that a strategy is just a move function, but that the result of the move is a monadic position value.
If we use the \texttt{IO} monad, we can represent a human player as a strategy. If we use a monad in the \texttt{MonadRandom} type-class, we can represent a random strategy. (Though, typically not completely random, of course.)

\begin{code}
module GameTheory.PoGa.Game
       (Player(First, Second),
        Game(..),
        Winner(..),
        terminal,
        Position(choices, winner),
        playGame,
        playTournament,
        opponent)
        where

data Player = First | Second
  deriving (Eq, Show)
           
type Strategy m p = p -> m p

data Winner = Neither | Both | Only Player
  deriving (Eq, Show)


newtype Game p = Game {position :: p}


-- The typeclass Position provides a very abstract view of
-- positional games.
class Position p where
  choices :: p -> [p]     -- available branches at position p
  winner :: p -> Winner   -- Nothing on non-leaf nodes
  terminal :: p -> Bool   -- Can a move be made? (For instance if the board is full,
                                                  the position is terminal)

opponent :: Player -> Player
opponent First = Second
opponent Second = First


playGame :: (Position p, Monad m) =>
  Game p -> Strategy m p -> Strategy m p -> m Winner
playGame (Game pos) firstStrategy secondStrategy
  | terminal pos = return $ winner pos
  | otherwise = do
      pos' <- firstStrategy pos
      playGame (Game pos') secondStrategy firstStrategy
      
playTournament :: (Monad m, Position p) =>
  Int -> Game p -> Strategy m p -> Strategy m p -> m [Winner]
playTournament n game fststrat sndstrat = do
  let g = playGame game fststrat sndstrat
  ws <- sequence $ replicate n g
  return ws
\end{code}

\subsubsection{SetGame.hs}

A typeclass works well as a general interface, but we need to have a concrete data-type to represent our games.

For many purposes, the following one works quite well. When it doesn't, a user of the library can always define his own data-type and use most of the functionality provided by the \texttt{PoGa} library, as long as the data-type is an instance of the \texttt{Position} typeclass, as described above.

\begin{code}

-- A module to use when your game is conveniently
-- definable by sets.

module GameTheory.PoGa.SetGame
       (SetGame(..), makeMove, fromWinningSets)
       where

import Data.Set as Set
import GameTheory.PoGa.Game

data SetGame v = SetGame {board :: Set v,               -- all vertices of board
                          turn :: Player,               -- who's turn is it?
                          firstChoices :: Set v,        -- vertices that First occupy
                          secondChoices :: Set v,       -- vertices that Second occupy
                          firstWin :: Set v -> Bool,    -- is First winner?
                          secondWin :: Set v -> Bool}   -- is Second winner?
                 
                 
instance Show v => Show (SetGame v) where
  show sg = show $ board sg

-- All unoccupied vertices
availableVertices :: Ord v => SetGame v -> Set v
availableVertices sg =  
  ((board sg) \\ (firstChoices sg)) \\ (secondChoices sg)

-- Current player makes move corresponding to the given vertex
makeMove :: Ord v => SetGame v -> v -> SetGame v
makeMove sg vtx = 
  case turn sg of
    First -> sg{firstChoices = Set.insert vtx (firstChoices sg),
                turn = Second}
    Second -> sg{secondChoices = Set.insert vtx (secondChoices sg),
                 turn = First}

-- construct a game given a board and winning sets
fromWinningSets :: (Ord v) =>
  Set.Set v -> Set.Set (Set.Set v) -> Set.Set (Set.Set v) -> SetGame v
fromWinningSets board wssFirst wssSecond = 
  SetGame {board = board,
           turn = First,
           firstChoices = Set.empty,
           secondChoices = Set.empty,
           firstWin = win wssFirst,
           secondWin = win wssSecond}
  where
  -- Make a win function from a set of winning sets
  win :: (Ord v) => Set.Set (Set.Set v) -> Set.Set v -> Bool
  win wss s =
    or [ws `Set.isSubsetOf` s | ws <- Set.toList wss]


-- All possible moves that can be made for the given game.
instance Ord v => Position (SetGame v) where
  choices sg = 
    Set.toList $ Set.mapMonotonic (makeMove sg) (availableVertices sg)
  winner sg =
    case (firstWin sg $ firstChoices sg,
          secondWin sg $ secondChoices sg) of
      (True, False) -> Only First
      (False, True) -> Only Second
      (False, False) -> Neither
      (True, True) -> Both
  terminal sg =
    (Set.null $ availableVertices sg) || (winner sg /= Neither)

\end{code}

\subsubsection {Strategies.hs}

In this section, we list the strategy submodule of \texttt{PoGa}.
It contains implementations of minimax with alpha-beta pruning as well as UCT.
The implementations of these were discussed in less detail in sections \ref{sec:alpha_beta} and \ref{sec:uct}, respectively.


As mentioned, strategies take a position to a monadic position, where the monad typeically is in \texttt{MonadRandom} or is the \texttt{IO} monad.

A strategy representing a human player would require some kind of user interface, which is not provided here.
It is provided in the module named \texttt{GraphicalGame}, and the required user interfaces are specified for the cases of arbitrarily sized Tic-Tac-Toe and Hex games.

Since it is not an important part of this paper, that entire module is left out of this section for space reasons.
However, the interested reader can download them from the repository (TODO: insert repository address here).




\begin{code}


odule GameTheory.PoGa.Strategy
       (Strategy,
        perfectStrategyFirst,
        perfectStrategySecond,
        randomStrategy,
        mctsStrategyFirst,
        mctsStrategySecond,
        MCTSNode (..))
       where


import Data.Function (on)
import GameTheory.PoGa.Game as Game
import qualified Data.Set as Set
import Data.List (find, sortBy, maximumBy, minimumBy)
import qualified Control.Monad.Random as Random

type Strategy m p = p -> m p

-- ~~~~~~~~~~ Exports ~~~~~~~~~~~~~~~~~~~~~~~~~~~~~~~~~

perfectStrategyFirst :: (Monad m, Position p) => p -> m p
perfectStrategyFirst pos = return $ perfectStrategy First pos
perfectStrategySecond :: (Monad m, Position p) => p -> m p
perfectStrategySecond pos = return $ perfectStrategy Second pos

-- smaple uniformly from all choices
randomStrategy :: (Random.MonadRandom m, Position p) => p -> m p
randomStrategy pos = Random.fromList [(c,1) | c <- choices pos]


-- ~~~~~~~~~~~~~~~~~~~~~~~~~~~~~~~~~~~~~~~~~~~~~~~~~~~~

prunedMaximumBy :: (a -> a -> Ordering) -> (a -> Bool) -> [a] -> a
prunedMaximumBy compare isMax xs = 
  case find isMax xs of
    Just x -> x
    Nothing -> maximumBy compare xs


alternatingStrategy :: Position p =>
  (Winner -> Winner -> Ordering) -> Player -> p -> Winner
alternatingStrategy compareWinners player pos
  | terminal pos = winner pos
  | otherwise =
      let ws = map (alternatingStrategy (flip compareWinners) (opponent player))
                   (choices pos) in
      prunedMaximumBy compareWinners ((==) (Only player)) ws
    

-- first is 'maxi' since Only First is greatest, in compareWinners
perfectStrategy:: Position p => Player -> p -> p
perfectStrategy player pos = 
  let cs = choices pos
      ws = map (alternatingStrategy (flip compareWinners) (opponent player)) cs
      wcs = zip ws cs in
  snd $ prunedMaximumBy (compareWinners `on` fst) ((==) (Only player) . fst) wcs
  where
    winnerValue w
      | w == (Only player) = 1
      | w == (Only $ opponent player) = -1
      | otherwise = 0
    compareWinners w w' = compare (winnerValue w) (winnerValue w') 




-- ~~~~~~~~~~~~~~~~~~~~~~~~~~~~~~~~~~~~~~~~~~~~~~~~~~~~~~~~~~~~~~~~~~~~~~~~~~
-- ~ Monte Carlo Tree Search (MCTS) strategy and supporting infrastructure
-- ~~~~~~~~~~~~~~~~~~~~~~~~~~~~~~~~~~~~~~~~~~~~~~~~~~~~~~~~~~~~~~~~~~~~~~~~~~

type Score = Double
data MCTSNode p = Unexplored p | Explored p Int Score [MCTSNode p]


instance Position p => Position (MCTSNode p) where
  terminal (Unexplored pos) = terminal pos
  terminal (Explored pos _ _ _) = terminal pos
  choices (Unexplored pos) = map Unexplored (choices pos)
  choices (Explored _ _ _ children) = children
  winner (Unexplored pos) = winner pos
  winner (Explored pos _ _ _) = winner pos

compareChildren node a@(Explored _ _ _ _) b@(Explored _ _ _ _) =
  compare (reconScore node a) (reconScore node b)
  where
  cExp :: Double  
  cExp = 1.0 / (sqrt 2.0) -- amount of exploration
  reconScore :: (Position p) => MCTSNode p -> MCTSNode p -> Score
  reconScore parent@(Explored _ vcp sp _) child@(Explored _ vcc sc _) =
    ( sc / (fromIntegral vcc) ) +
    cExp * sqrt (  2.0*(log $ fromIntegral vcp) / (fromIntegral vcc)  )
compareChildren _ (Unexplored _) (Unexplored _) = EQ
compareChildren _ (Explored _ _ _ _) (Unexplored _) = LT
compareChildren _ (Unexplored _) (Explored _ _ _ _) = GT 


-- This function is only defined for non-terminal nodes
popBestChild :: Position p => MCTSNode p -> (MCTSNode p, [MCTSNode p])
popBestChild node@(Explored pos visitCount score children) = 
  popMaximumBy (compareChildren node) children
popBestChild (Unexplored _) = error "popBestChild: un-explored node given"



mctsStrategy :: (Position p, Random.MonadRandom m) =>
  (p -> Score) -> Int -> MCTSNode p -> m (MCTSNode p)
mctsStrategy leafValue 0 node = do
  let (c, _) = popBestChild node
  return c
mctsStrategy leafValue numSteps node = do
  (_, node') <- explore leafValue node
  mctsStrategy leafValue (numSteps-1) node'



explore :: (Position p, Random.MonadRandom m) =>
  (p -> Score) -> MCTSNode p -> m (Score, MCTSNode p)
explore leafValue (Unexplored pos) = do
  s <- recon leafValue pos
  return $ (-s, Explored pos 1 s (map Unexplored $ choices pos))
explore leafValue node@(Explored pos visitCount score children)
  | terminal node = do
      let s = leafValue pos
      return $ (-s, Explored pos (visitCount+1) s [])
  | otherwise = do
      let (c, cs) = popBestChild node
      (s, c') <- explore leafValue c
      return $ (-s, Explored pos (visitCount+1) (score+s) (c':cs))

recon :: (Random.MonadRandom m, Position p) => (p -> Score) -> p -> m Score
recon leafValue pos
  | terminal pos = return $ leafValue pos
  | otherwise = do
      c <- Random.fromList [(c,1) | c <- choices pos]
      s <- recon leafValue c
      return $ -s

-- helper function: takes a non-empty list and
-- extracts it's maximum
popMaximumBy :: (a -> a -> Ordering) -> [a] -> (a, [a])
popMaximumBy _ [] = error "popMaximumBy: empty list"
popMaximumBy _ [x] = (x, [])
popMaximumBy cmp (x:xs) = 
  let (m, xs') = popMaximumBy cmp xs in
  if cmp m x == LT then (x, m:xs') else (m, x:xs')





valueFirst :: Position p => p -> Score
valueFirst pos =
  case winner pos of
    Neither -> 0.0
    Both -> 0.0
    Only First -> -1.0
    Only Second -> 1.0

valueSecond :: Position p =>  p -> Score
valueSecond = negate . valueFirst


mctsStrategyFirst :: (Position p, Random.MonadRandom m) =>
  Int -> MCTSNode p ->  m (MCTSNode p) 
mctsStrategyFirst = mctsStrategy valueFirst
mctsStrategySecond :: (Position p, Random.MonadRandom m) =>
  Int -> MCTSNode p ->  m (MCTSNode p)
mctsStrategySecond = mctsStrategy valueSecond




-- ~~~~~ Tests ~~~~~~~~~~~~~~~~~~~~~~~~~~~~



test_popMaximumBy :: Ord a => a -> [a] -> Bool
test_popMaximumBy x xs =
  let (m, xs') = popMaximumBy compare (x:xs) in
  length (x:xs) == length (m:xs') && m == maximumBy compare (x:xs)


\end{code}


% bibliography
% \bibliographystyle{abbrv}
\bibliographystyle{plainnat}
\bibliography{simple}
\begin{thebibliography}{9}


\bibitem[Beck, 2008]{beck08}
József Beck, Combinatorial Games: Tic-Tac-Toe Theory
\emph{Cambridge University Press, 2008}

\bibitem[Demaine, Hearn, 2008]{demaine_hearn08}
Erik D. Demaine, Robert A. Hearn, Playing Games with Algorithms: Algorithmic Combinatorial Game Theory
\emph{http://arxiv.org/abs/cs/0106019 (arXiv:cs/0106019)}

\bibitem[MCTS Survey, 2012]{mcts_survey12}
A Survey of Monte Carlo Tree Search Methods
\emph{IEEE Transactions on AI and computational intelligence in games, Vol. 4, No 1, March 2012}


\bibitem[Russell, Norvig, 2012]{aimodernapproach}
Stuart Russell, Peter Norvig, Artificial Intelligence: A Modern Approach (3rd Edition) 
\emph{Prentice Hall Series in Artificial Intelligence, 2009}


\end{thebibliography}


\end{document}

