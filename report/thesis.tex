\documentclass[12pt]{article}
\usepackage{amsmath}
\usepackage{geometry}
\usepackage{amsfonts}
\usepackage{amssymb}
\usepackage{natbib}
\usepackage{verbatim}
\usepackage{listings}
\usepackage{array}
\usepackage{alltt}
\usepackage{lscape}
\usepackage{enumerate}
\usepackage[utf8]{inputenc}
\usepackage{fancyvrb}
\usepackage{tikz}
\usepackage{multirow}
\usepackage[nottoc,numbib]{tocbibind}
\DefineVerbatimEnvironment{code}{Verbatim}{fontsize=\small}
\DefineVerbatimEnvironment{datalisting}{Verbatim}{fontsize=\scriptsize}


\begin{document}
\lstset{language=Haskell,basicstyle=\ttfamily\small,breaklines=true}

\newtheorem{theorem}{Theorem}[section]
\newtheorem{lemma}[theorem]{Lemma}
\newtheorem{proposition}[theorem]{Proposition}
\newtheorem{definition}[theorem]{Definition}
\newtheorem{corollary}[theorem]{Corollary}
\newtheorem{conjecture}[theorem]{Conjecture}

\newenvironment{proof}[1][Proof]{\begin{trivlist}
\item[\hskip \labelsep {\bfseries #1}]}{\end{trivlist}}

% we want our definitions numbered!
%\newenvironment{definition}[1][Definition]{\begin{trivlist}
%\item[\hskip \labelsep {\bfseries #1}]}{\end{trivlist}}

\newenvironment{notation}[1][Notation]{\begin{trivlist}
\item[\hskip \labelsep {\bfseries #1}]}{\end{trivlist}}

\newenvironment{example}[1][Example]{\begin{trivlist}
\item[\hskip \labelsep {\bfseries #1}]}{\end{trivlist}}

\newenvironment{remark}[1][Remark]{\begin{trivlist}
\item[\hskip \labelsep {\bfseries #1}]}{\end{trivlist}}

%%%%%%%%%%%%%%%%%%%%%%%%%%%%%%%
% some custom commands
%%%%%%%%%%%%%%%%%%%%%%%%%%%%%%%

\newcommand{\qed}{\nobreak \ifvmode \relax \else      % qed sign used to mark end of proofs
      \ifdim\lastskip<1.5em \hskip-\lastskip         
      \hskip1.5em plus0em minus0.5em \fi \nobreak     
      \vrule height0.75em width0.5em depth0.25em\fi}  
\newcommand{\fg}{\mathfrak M}                         % set of fin. gen. modules

\newcommand{\isomarrow}{\overset\sim\to}              % isomorphism arrow

%%%%%%%%%%%%%%%%%%%%%%%%%%%%%%%

\title{Exact and Monte-Carlo algorithms for combinatorial games}
\date{February 24, 2014}
\author{Anders Leino}
\maketitle
\begin{abstract}
This thesis concerns combinatorial games and algorithms that can be used to play them.
Basic definitions and results about combinatorial games are covered, and an implementation of the minimax algorithm with alpha-beta pruning is presented.
Following this, we give a description and implementation of the common UCT variant of MCTS (Monte-Carlo tree search).
Then, a framework for testing the behavior of UCT as first player, at various numbers of iterations ~(namely 2,7, \dots 27), versus minimax as second player, is described.
Finally, we present the results obtained by applying this framework to the 2.2 million smallest non-trivial positional games having winning sets of size either 2 or 3.
It is seen that on almost all different classifications of the games studied, UCT converges quickly to near-perfect play.
\end{abstract}
\pagebreak
\renewcommand{\abstractname}{Exakta och Monte-Carlo algoritmer för kombinatoriska spel}
\begin{abstract}
Denna rapport angår kombinatoriska spel och algoritmer som kan användas för att spela dessa.
Grundläggande definitioner och resultat som angår kombinatoriska spel täcks, och en implementation av minimax-algoritmen med alpha-beta beskärning ges.
Detta följs av en beskrivning samt en implementation av UCT varianten av MCTS (Monte-Carlo tree search).
Sedan beskrivs ett ramverk för att testa beteendet för UCT som första spelare, vid olika antal iterationer ~(nämligen 2, 7, \dots 27), mot minimax som andra spelare.
Till sist beskrivs resultaten vi funnit genom att använda detta ramverk för att spela de 2,2 miljoner minsta icke triviala positionella spelen med vinstmängder av storlek antingen 2 eller 3.
Vi finner att, för nästan alla olika klassificeringar av spel vi studerar, så konvergerar UCT snabbt mot nära perfekt spel.
\end{abstract}
\pagebreak
\tableofcontents
\pagebreak
\section{Introduction}
\label{sec:introduction}
\subsection{What is a combinatorial game?}

Game theory in general is a large field.
To give an idea about what kinds of games people have studied, here are two orthogonal categorizations.

Firstly there is the \emph {chance versus skill} aspect. There are games of pure chance and no skill, games of some chance and some skill, and games of pure skill.
Combinatorial games are games of pure skill and no chance, and Poker is a game of some chance and some skill.

Secondly, there are games of \emph{complete information} and games of \emph{incomplete information}.
Poker is a game of incomplete information, and combinatorial games are games of complete information.

We will be concerned only with combinatorial games played by two players.

\subsubsection{Examples of combinatorial games}

The game of \emph{Nim} is a good example of a combinatorial game.
Nim is a game for two players, and can be played using nothing more than a bunch of beans (or other small objects).
The game is played by arranging the beans in a number of heaps.
The players take turns performing moves. A move is made by selecting a heap, and then taking a number of beans from that heap only. The winner is the last person to take a bean.

Another very popular combinatorial game is \emph{checkers}. (Also known as \emph{draughts}.)
Checkers is played by two players on a checkerboard of some size, say 8-by-8 or 12-by-12 squares.
Each player starts out with pieces of a given color in a certain pattern on his side of the board.
The players then take turns performing moves on their own pieces.
A move consists in selecting a piece of ones own color, and moving it to an adjacent location.
If the opponent has a piece in such a location, it may be captured by jumping over it to the next adjacent location in the same direction, in which case it is removed from the board. If a move captured an opponent piece, it may be extended in case it can capture more of the opponents pieces in the same way, in a subsequent move. A piece may also become a so-called ``king'', in which case it can move in more ways.

A good introduction to combinatorial game theory is \citep{winning_ways}.

\subsection{Advances in combinatorial game theory}

For a given combinatorial game, it is quite easy to convince oneself that it would be possible, in principle, to exhaustively enumerate every single possible play of that game.
In a few of those plays, one player may act ideally in all situations, and in even fewer, both players will act ideally in all situations.

However, playing by investigating all possible plays will quickly introduce a player to the concept of \emph{combinatorial chaos} -- the number of possible ways of playing the game, though finite, can be enormous.
Therefore it is natural to wonder if one can find an algorithm with reasonable computational complexity, to play a given game perfectly.

Several results along these lines are presented in \citep{demaine_hearn08}. For instance: on page 10 we learn that checkers has been determined to be a draw if both players play perfectly, but that playing perfectly is a hard computational problem.

\subsubsection{Nim-like games and Sprague-Grundy theory}

It should be mentioned that, in spite of combinatorial chaos, some very nice results have been found for a certain class of games -- so-called ``Nim-like'' games.
In the 1930's, Sprague and Grundy both independently showed that \emph{impartial games} are equivalent to Nim.
An impartial game, roughly, is a game where the players can both make the same moves, i.e. where the allowable moves at any moment only depends on what configuration the game is in, and not on who's turn it is to move.
This is a big discovery because Nim can be considered a solved game -- at any moment in a game of Nim, it is possible to quickly say which player would win in perfect play.

\subsubsection{Probabilistic approach to game theory}

An interesting approach to tackling combinatorial chaos by means of probabilistic methods is presented in \citep{beck08}.


\section{Positional games}
\label{sec:positional_games}
\subsection{Definitions}

To define a \emph{positional game}, or \emph{strong game}, or simply \emph{game} from now on, we need a few things.
First of all, we need a ``board'', $V$, which is just a finite set, as well as a collection of \emph{winning sets}, $\mathcal F \subset \mathcal P(V)$.
The tuple $(V,\mathcal F)$ constitutes a \emph{hypergraph}.
The elements of the set $V$ are sometimes called the \emph{vertices} and the elements of the set $\mathcal F$ are sometimes called the \emph{hyperedges} of the hypergraph $(V,\mathcal F)$.

\begin{remark}
We often refer to $(V,\mathcal F)$ as the game.
\end{remark}

The idea is that two players, called $\emph{First}$ and $\emph{Second}$ take turns coloring uncolored vertices of the board.
Initially, the entire board, $V$, starts out with all vertices uncolored.
The object of the game is to be the first to color an entire winning set. The players are named as they are because player First has the benefit of getting the first move.

Note that a vertex which has already been colored cannot be colored again.
The word \emph{play} is meant to represent an instance of a correctly played game from start to finish.
A given point of the play is called a \emph{position} of the board.
More precisely, a position is a (partial or complete) two-coloring of $V$.
A \emph{draw} happens when the board is fully occupied, yet neither player occupies completely a winning set.

As an example of a game, look at table \ref{tab:ex_positional}. It shows an entire play for the game Hex, played on a 3x3 board.
The goal is to connect one horizontal side with the opposing one. First wins because he connects the two sides before Second.
\begin{center}
\def\arraystretch{5.5}
\begin{table}
\begin{tabular}{l c r}
  \def\svgwidth{0.3\columnwidth} \input{hex_board_3x3_positional_game_01.pdf_tex} &
  \def\svgwidth{0.3\columnwidth} \input{hex_board_3x3_positional_game_02.pdf_tex} &
  \def\svgwidth{0.3\columnwidth} \input{hex_board_3x3_positional_game_03.pdf_tex} \\
  \def\svgwidth{0.3\columnwidth} \input{hex_board_3x3_positional_game_04.pdf_tex} &
  \def\svgwidth{0.3\columnwidth} \input{hex_board_3x3_positional_game_05.pdf_tex} &
  \\
\end{tabular}
\caption{3x3 Hex, First wins}
\label{tab:ex_positional}
\end{table}
\end{center}

\subsubsection{Reverse games}

The \emph{reverse} of a given game is obtained if the desired outcome is to avoid occupying completely the winning sets from $\mathcal F$.
In table \ref{tab:ex_reverse}, we can see a play of reverse Hex 3x3. This time, Second wins. 

\begin{center}
\def\arraystretch{5.5}
\begin{table}
\begin{tabular}{l c r}
  \def\svgwidth{0.3\columnwidth} \input{hex_board_3x3_reverse_game_01.pdf_tex} &
  \def\svgwidth{0.3\columnwidth} \input{hex_board_3x3_reverse_game_02.pdf_tex} &
  \def\svgwidth{0.3\columnwidth} \input{hex_board_3x3_reverse_game_03.pdf_tex} \\
  \def\svgwidth{0.3\columnwidth} \input{hex_board_3x3_reverse_game_04.pdf_tex} &
  \def\svgwidth{0.3\columnwidth} \input{hex_board_3x3_reverse_game_05.pdf_tex} &
  \def\svgwidth{0.3\columnwidth} \input{hex_board_3x3_reverse_game_06.pdf_tex} \\
  \def\svgwidth{0.3\columnwidth} \input{hex_board_3x3_reverse_game_07.pdf_tex} &
  \def\svgwidth{0.3\columnwidth} \input{hex_board_3x3_reverse_game_08.pdf_tex} &
  \def\svgwidth{0.3\columnwidth} \input{hex_board_3x3_reverse_game_09.pdf_tex} \\
\end{tabular}
\caption{Reverse 3x3 Hex, Second wins}
\label{tab:ex_reverse}
\end{table}
\end{center}

\subsubsection{Weak games}

In the above definition of a a game, both players (First and Second), strive to occupy the same winning sets, given by $\mathcal F$. A player might be interested in settling for a draw. (For instance, if Second knows that he cannot win.) Thus, we have two players: one player is the \emph{Maker}, and one is the \emph{Breaker}.

We say that Maker wins if he manages to occupy completely one of the winning sets in $\mathcal F$, and Breaker wins if he manages to prevent maker.
The notion of who is ``first to win'' is moot: a player either wins or doesn't win.
This game is called a the \emph{weak version} of the original game, or the corresponding \emph{Maker-Breaker} game.
Note that a draw is impossible in a Maker-Breaker game.

Table \ref{tab:ex_weak} shows an example of weak 3x3 Hex being played, with First as Maker and Second as Breaker. First (Maker) wins, since he manages to connect the two horizontal sides.
\begin{center}
\def\arraystretch{5.5}
\begin{table}
\begin{tabular}{l c r}
  \def\svgwidth{0.3\columnwidth} \input{hex_board_3x3_weak_game_01.pdf_tex} &
  \def\svgwidth{0.3\columnwidth} \input{hex_board_3x3_weak_game_02.pdf_tex} &
  \def\svgwidth{0.3\columnwidth} \input{hex_board_3x3_weak_game_03.pdf_tex} \\
  \def\svgwidth{0.3\columnwidth} \input{hex_board_3x3_weak_game_04.pdf_tex} &
  \def\svgwidth{0.3\columnwidth} \input{hex_board_3x3_weak_game_05.pdf_tex} &
  \\
\end{tabular}
\caption{Weak 3x3 Hex, First (Maker) wins}
\label{tab:ex_weak}
\end{table}
\end{center}

\subsubsection{Reverse weak games}

The notion of a \emph{reverse weak game} should now be intuitively clear.
The idea is to start with a game, get the corresponding weak game and then reverse that.
However, since a weak game is technically a not a game, we should give an explicit definition of what reverse means in reference to a weak game.
Suppose we have a game with the hypergraph $(V,\mathcal F)$.
The corresponding weak game has a Maker and a Breaker. The reverse weak game has a player trying to avoid making, and a player trying to avoid breaking. That is: we have an \emph{Avoider} and an \emph{Enforcer}.
Avoider tries to avoid occupying completely a wining set from $\mathcal F$, and Enforcer tries to prevent Avoider from doing so, that is, tries to enforce Avoider into occupying completely a winning set.

In table \ref{tab:ex_reverse_weak}, we can see an example of reverse weak 3x3 Hex being played. Second (Enforcer) wins, since First has connected the two vertical edges.
\begin{center}
\def\arraystretch{5.5}
\begin{table}
\begin{tabular}{l c r}
  \def\svgwidth{0.3\columnwidth} \input{hex_board_3x3_reverse_weak_game_01.pdf_tex} &
  \def\svgwidth{0.3\columnwidth} \input{hex_board_3x3_reverse_weak_game_02.pdf_tex} &
  \def\svgwidth{0.3\columnwidth} \input{hex_board_3x3_reverse_weak_game_03.pdf_tex} \\
  \def\svgwidth{0.3\columnwidth} \input{hex_board_3x3_reverse_weak_game_04.pdf_tex} &
  \def\svgwidth{0.3\columnwidth} \input{hex_board_3x3_reverse_weak_game_05.pdf_tex} &
  \def\svgwidth{0.3\columnwidth} \input{hex_board_3x3_reverse_weak_game_06.pdf_tex} \\
  \def\svgwidth{0.3\columnwidth} \input{hex_board_3x3_reverse_weak_game_07.pdf_tex} &
  \def\svgwidth{0.3\columnwidth} \input{hex_board_3x3_reverse_weak_game_08.pdf_tex} &
  \def\svgwidth{0.3\columnwidth} \input{hex_board_3x3_reverse_weak_game_09.pdf_tex} \\
\end{tabular}
\caption{Reverse weak 3x3 Hex, Second (Enforcer) wins}
\label{tab:ex_reverse_weak}
\end{table}
\end{center}

\subsection{The game tree}
\label{subsec:gametree}
A good conceptual tool when reasoning about positional games is the so-called \emph{game tree}, corresponding to a given game.

The root node of the game tree is the starting position of the game.
That is to say, it corresponds to the empty board of the game.
The children of the root node are all possible positions of the board after First has made his first move.
The children of \emph{those} nodes are the possible positions after Second has made his move, and so on.
Since we are only considering games played on finite boards, the game tree is finite as well, but can be extremely large.

In table \ref{tab:ex_gametree} you can see an example of a portion of the game tree for 3x3 Tic-Tac-Toe, below a certain position which has been chosen as the root node. The large white squares are not part of the tree; they are only in the picture to provide visual balance.
The positions for a move are considered from left to right, bottom to top, on the game board. The first free position encountered when traversing the board in this manner is chosen.

\begin{center}
\def\arraystretch{5.5}
\begin{table}
\def\svgwidth{\columnwidth} \input{tic-tac-toe_3x3_gametree.pdf_tex}
\caption{A part of the game tree for 3x3 Tic-Tac-Toe}
\label{tab:ex_gametree}
\end{table}
\end{center}


\section{Optimal play}
\label{sec:optimal_play}
\subsection{Introduction}

In this section, we will introduce an algorithm called the minimax algorithm which can play any positional game (and more genral games) in an optimal manner.

Naturally, any such algorithm will suffer from performance issues.
Nevertheless, there is one common optimization called alpha-beta pruning, which we will also describe.

The natural structure on which algorithmic play takes place is the game tree, as described in section \ref{subsec:gametree}.
A playing algorithm can then be seen as a search algorithm on the game tree.


\subsection{Optimal play}

What do we mean by ``optimal play''?

In this section, we will give some pretty intuitive neccesary conditions.

Clearly, a player cannot play optimally if he squanders an oppurtinity to win.
More precisely; if current position is a win for the moving player, then he must make a choice which is also a win for him.

Furthermore, if the moving player does not have any opportunity to win in the given position (i.e. none of the leaf-nodes in the tree attached to the current node contains a winning node for the moving player), but if he does have an opportinity to tie, he must still have an opportunity to tie after he makes his move.

Thus, if a player can play as above, then since the game can't go on forever, he will win if he can win, and if he can't win but he can tie, he will do that.


TODO: Standard definition of optimal play?


\subsection{The minimax algorithm}

In this section, we will see how the considerations in the previous sections guide us to a pretty intuitive algorithm which leads to optimal play.

It is intuitively clear that it is always possible to play as described in the previous section.

The key in order to find an explicit algorithm is to extend the notion of First player win to be not just for leaf nodes.

In terms of the game tree, what might it mean for any given position, not just a leaf node, to be a First player win?
The definition is inductive.

In the base case, i.e. we have a leaf node, the definition of winning node is clear from the rules of the game.
If we are not on a leaf node, we break the definition up into two cases:

\begin{itemize}
  \item Case 1: It is Firsts turn to move.
    In this case, the position is a First win it has a child which is a first win.
  \item Case 2: It is Seconds turn to move.
    In this case, the position is a First win if, for all of it's children, they have children which is a First win.
\end{itemize}

The notion of a position being a Second win is defined similarly.

It is easy to see that if a position is a First win, it cannot be a Second win, and vice-versa.
This does not mean that a position must be either a First win or a Second win; and we call such positions Neither win positions.

We have now defined a coloring of any game tree, with three different colors: First win, Second win and Neither win.

The following results follow directly from the definition of optimal play:

If the game is in a First win position, then the play won't change color if First plays optimally.
(And similarly for Second.)

If the game is in a Neither win position and First plays optimally, then the game will never be in a Second win color, but it might turn into a First win color unless Second also plays optimally.
(And similarly for Second.)

Finally: if both players play optimally, then the color of the game won't change.

From the above results, the central theorem, which corresponds to the so-called minimax algorithm follows easily:
The color of a node is the same as the color of the leaf node which results when both players play optimally.
(This is well defined because of the above corollary.)

To get something a bit more operational, we define an ordering on the set of colors:

\[
  \text{Second win} < \text{Neither win} < \text{First win}
\]

It is clear that First plays optimally if he at all times makes choices with maximum color.
Similarly, Second plays optimally if he at all times makes choices with minimum color.

Thus, to find the color of a position in which it's Firsts time to move, we look at each of it's children in order.
If any child is a First win, we know that our node is a First win, and can stop searching.

To find the color of a position in which it's Seconds time to move, we look at each of it's children in order.
If any child is a Second win, we know that the our node is a Second win, and can stop searching.

This leads to the following mutually recursive definition of color for a node, which can me used to play optimally:

\begin{code}
  winner play@(Play First _) = 
    max $ map winner $ options play
  winner play@(Play Second _) = 
    min $ map winner $ options play
\end{code}


It is clear that, if we have this function, it is trivial to play optimally according to the above definition.

The actual code used to run the experiments later on is not quite this simple: as mentioned we need to do some optimizations.

\subsection {Alpha-Beta Pruning}

The idea for this optimization is quite simple: suppose that, as we are evaluating max function playing as First, we run into a First win.
We then know that we can stop searching because we cannot do any better than that.

Similarly if Second is evaluating the min function, if he runs into Second win, he can stop looking since he cannot do any better than that in any case.

In other words, we are ``pruning'' the game tree as we search it.


This is a specific case of alpha-beta pruning, but for the case where the leaf node can have a bigger set of values than just First win, Second win and Neither win is completely analogous.

The implementation is particularly easy in Haskell. We can even make it look exactly like the implementation above, but we need to take some special care when writing min and max so that they are lazy and that they know what the absolute minimum is (Second win) and what the absolute maximum is (First win) and can therefore prune.

The lazyness means that Haskell can capture the semantics of the list of winners in a given position, without actually computing the entire list.
The ``early out'' part is just a form of pruning on the list on which the min and max operate, so that for some lists we will get to the answer without inspecting the entire list (meaning, in Haskell, to compute all of the elements in the list).

Here is prunedMin:

\begin{code}
  prunedMin ws = 
    case find (Only Second) of
      Nothing -> min ws
      _ -> Only First
\end{code}

This is a lazy function since find is lazy: therefore, not all of ws might be computed - it depends on what find finds.
If we have looked trough the entire list of winners and not found Only Second, we give up and use the regular min function.

The implementation for prunedMax is similar:

\begin{code}
  prunedMax ws = 
    case find (Only First) of
      Nothing -> max ws
      _ -> Only First
\end{code}


Now we can rewrite our pruned minimax algorithm so that it is very similar to the one in the previous section:

\begin{code}
  winner play@(Play First _) = 
    prunedMax $ map winner $ options play
  winner play@(Play Second _) = 
    prunedMin $ map winner $ options play
\end{code}


Again, the winner function is the interesting part. If one has a winner function, it is easy to fill in the details required to derive a completely generic strategy.


\section{MCTS}
\label{sec:mcts}
\subsection {Introduction}

Monte-Carlo tree search (MCTS) has become an umbrella term for a class of related algorithms for searching trees probabilistically.
This applies directly to games if we decide to search the game tree.

In this section, we will introduce an MCTS algorithm known as UCT.
We will mostly follow the exposition in \citep{mcts_survey12}, chapter 3.

\subsection {MCTS in general}

MCTS studies the game tree as follows.
It keeps a record of a subtree of the game-tree containing the nodes that the algorithm has visited so far.
It also keeps some extra information about each node, which is supposed to represent an approximation of the ``value'' of that node.
The idea is to somehow find a good \emph{expandable node} (meaning that it has unvisited children) in the visited part of the game tree, and then to make an excursion from that node, which means doing a quicker kind of search from the node, in order to estimate the value of the node. The information gleaned from this excursion will then contribute to the algorithms knowledge of the game tree.

It is assumed that we have a (reasonably efficient) function that lets us determine the value of a leaf node.
Here is a sketch of the steps that will make up our algorithm:

\begin{itemize}
\item \emph{Selection}: find a suitable expandable explored node, and select one of it's child nodes.
\item \emph{Exploration}: run a simulation from the newly found child node and return a \emph{score}.
\item \emph{Backpropagation}: use the score found in the previous step to update the visited tree in an appropriate way.
\end{itemize}
These steps are iterated a number of times in order to make a single move. Each iteration yields a more complete and refined knowledge of the game tree, thanks to the backpropagation step. In order to subsequently make a move, a sort of ``single level'' version of the selection step is carried out.
Note that there are variants of this algorithm which expand and explore multiple nodes instead of just one, but the principle is the same otherwise.
Note also that this algorithm is far from complete. There are various appropriate ways of performing each of these steps, depending on the situation.
The next section describes one of the possibilities: the UCT (Upper Confidence bounds for Trees) algorithm.

\subsection{The UCT algorithm}
\label{sec:uct}

In this section, we fill in each of the steps outlined in the previous section, for the special case of the UCT algorithm.

Each node $v$ in the explored part of the game tree has an attached score, which is just a real-valued number, say $S(v)$.


\subsubsection{The selection step}
\label{subsec:uct_selection_step}

Selection takes place in the explored part of the game tree, and can therefore use the score, $S$.
We repeatedly pick the ``best child'' of the current node, in the following sense.
If $v$ has children which have not been explored, then pick any of them as the best child.
If all children of $v$ have already been explored, then we pick a child, $v'$, which maximizes

\begin{equation}
\label{eq:uctnodevalue}
S(v') = \frac{Q(v')}{N(v')} + c\sqrt{\frac{2\ln{N(v)}}{N(v')}}
\end{equation} 
where $v'$ is a child of $v$, $N$ is the visit count and $Q$ is the accumulated score for a node (we will see later how to keep track of $Q$ and $N$, for a given node).
The parameter $c$ determines the amount of exploration. We will choose $c = 1 / \sqrt 2$ as per the comments in \citep[p. 9]{mcts_survey12}.

This selection process continues until we find either an unexplored node or run into a node without children (i.e. a leaf node), in which case we return that leaf node.

\subsubsection{The exploration step}

When we have found a node using the selection step, we will explore that node, which will yield a score.
If we are ``exploring'' a leaf node, the score will just be the value at the leaf node.
Otherwise, we are exploring an unexplored non-leaf node, and then simply search randomly from that node until we run into a leaf node, which we know how to evaluate a score for.
If the node has not previously been explored, it will finally be marked as explored, and it's visit count and score will be initialized as appropriate.

\subsubsection{The backup step}

When the exploration is done  we go back up the way we came, all the way to the root node.
As we go, we update visit count, $N$, and accumulate the score, $Q$.
We also make sure to alternate the sign of the score we use to accumulate as we go up the tree, since it represents the value for the player who's move it is at that node.

\subsection{An example}
In this section, we give an example of the UCT algorithm described above.
To keep things simple, we do only a single iteration.
However, in order to get a non-trivial iteration, we assume that five iterations have already been done, and do iteration number six.
\begin{center}
\def\arraystretch{5.5}
\begin{table}
\begin{tabular}{l}
  \def\svgwidth{\columnwidth} \input{mcts_example_before.pdf_tex}
\end{tabular}
\caption{Before the iteration}
\label{tab:mcts_iteration_before}
\end{table}
\end{center}
In table \ref{tab:mcts_iteration_before}, we have our starting point.
We must first carry out the selection step. As can be seen from the figure, there are two cases: one with $N=1$ and $Q=-1$, and one with $N=1$ and $Q=1$.
Let the choices on the second row of table \ref{tab:mcts_iteration_before} be $v_1$, $v_2$, $v_3$ and $v_4$, from left to right.
Using expression \ref{eq:uctnodevalue} with $c = 1/\sqrt{2}$, we get $S(v_1) = \sqrt{\ln{5}} - 1$ for the first choice, and $S(v_2) = S(v_3) = S(v_4) = \sqrt{\ln{5}} + 1$ for the other three choices.
So $v_2$, $v_3$ and $v_4$ all maximize $S$.
Since this node has no explored children, we do not need to calculate \ref{eq:uctnodevalue} -- we simply select the first child, i.e. the fourth node from the left, on the third row.
This node is the result of the selection step.

Now it is time to perform the exploration step.
This is just a random search from the selected node, until we hit a leaf node.
As can be seen from table \ref{tab:mcts_iteration_before}, there are two possible outcomes: either the seventh node on the fourth row, or the eight node on the fifth row.
Suppose that the outcome is the latter.
This node is a win for First.
We should now perform the backup step. We start with the score $1$, and alternate signs until we come back up to the selected node. Which gets initialized with $N=1$ and $Q=(-1)^2 = 1$.
The backup step is not done yet: we should continue all the way up to the root node, remembering to alternate the sign and adjust the accumulated scores and visit counts as we go.
The complete result after the backup step is shown in table \ref{fig:mcts_iteration_after}.
\begin{center}
\def\arraystretch{5.5}
\begin{table}
\begin{tabular}{l}
  \def\svgwidth{\columnwidth} \input{mcts_example_after.pdf_tex}
\end{tabular}
\caption{After the iteration}
\label{fig:mcts_iteration_after}
\end{table}
\end{center}

\subsection{UCT implementation}

It is assumed that we have a function, \texttt{value}, which can only be applied to terminal values and which gives a real number representing the value of a given node from the point of view of the player in turn (i.e. the opponent of the player who made the previous move). The value is $1$ if the position is a winning position for the player in turn, $-1$ if it is a losing position, and $0$ otherwise.
We also assume that we have a function named \texttt{choices}, which we can apply to a node in order to find the set of possible choices of nodes that the player in turn could move to. Finally, we assume that our nodes may carry ``MCTS data'' of the following format.
\begin{lstlisting}[frame=single]
data MCTSNodeData = MCTSNodeData { visitCount :: Int,
                                   score :: Score }
\end{lstlisting}
A node only carries \texttt{MCTSNodeData} if it has been explored by the \texttt{explore} function, below.
The \texttt{Score} type is just a synonym for a real number type, like \texttt{Float}. The function \texttt{getMCTSData} will return the MCTS data for a given node if it has any, and we can set it by means of \texttt{setMCTSData}.

We begin by looking at the \texttt{recon} function, which denotes the score \emph{from the point of view of the player in turn}, of a random search from the given position.
\begin{minipage}{\linewidth}
\begin{lstlisting}[frame=single]
recon position
  | Game.terminal position = return $ value pos
  | otherwise = do
      c <- Random.fromList [(c,1)
                           | c <- Game.choices position]
      s <- recon c
      return $ -s
\end{lstlisting}
\end{minipage}
It reads as follows: in case the given position is terminal, then we can return the \texttt{value} of that position. Otherwise we select a random child, apply \texttt{recon} to it, and get a score, \texttt{s}, back.
Note that $\texttt{s}$ represents the score from the point of view of the opponent of the player in turn, since \texttt{recon} was applied to a child of \texttt{position}. Therefore we must negate the score before we return it.

Next, we'll look at the \texttt{explore} function, which is the core of the algorithm. The return value of the function is a tuple of a score together with the explored node. Just like \texttt{recon}, the score it returns is relative to the player who's turn it is in the given node. Apart from the node to explore, it also takes \texttt{cExp}, which is just the $c$ parameter from expression~\ref{eq:uctnodevalue}.

\begin{minipage}{\linewidth}
\begin{lstlisting}[frame=single]
explore cExp node = 
  case getMCTSData node of
    Nothing -> do
      s <- if Game.terminal node
           then return $ value node
           else recon node
      return ( s,
               setMCTSData node $
               MCTSNodeData {visitCount = 1,
                             score = s} )
    Just (MCTSNodeData {visitCount = vc,
                        score = sc}) -> do
      case Game.terminal node of
        True -> do
          let s = value node
          return ( s,
                   setMCTSData node $
                   MCTSNodeData {visitCount = vc + 1,
                                 score = sc + s} )
        False -> do
          let (c, cs) = popBestChild cExp node
          (s, c') <- explore cExp c
          let s' = negate s
              node' = setChoices node (c':cs) in
            return ( s',
                     setMCTSData node' $
                     MCTSNodeData {visitCount = vc + 1,
                                   score = sc + s'} )
\end{lstlisting}
\end{minipage}
At the top level, the function is split up into two cases -- either our node has no MCTS data, or it does.
If the node does not have MCTS data, we obtain a score, \texttt{s}, in either of two ways: using the \texttt{value} function if the node is terminal, or else by applying the \texttt{recon} to the node, i.e. doing a random search.
In either case we get a score, and so we can return the explored version of our node.

In case the node has MCTS data, i.e. is explored, we again have two sub-cases: terminal or not terminal.
If the explored node is terminal, we again use the value function to obtain a score, and use that to update the MCTS data for the node.
If the explored node is not terminal, then we select the best child as described in section~\ref{subsec:uct_selection_step}, explore that child recursively with another call to \texttt{explore} and use the result to return an updated node.

One important ingredient is the \texttt{compareChildren} function. It takes a node and two children, \texttt{a} and \texttt{b}, of the node, and returns an ordering, which is just a type for representing ``less than'', ``equal to'' or ``greater than''.
It allows us to sort a set of children, and therefore to write \texttt{popBestChild} and \texttt{findBestMove} (below), with relative ease.
The function is just a straight encoding of the rules mentioned in \ref{subsec:uct_selection_step}.
\begin{lstlisting}[frame=single]
compareChildren cExp node a b =
  case (getMCTSData a, getMCTSData b) of
    (Just aData, Just bData) ->
      let Just parentData = getMCTSData node in
      compare (reconScore parentData aData) (reconScore parentData bData)    
      where
        reconScore :: MCTSNodeData -> MCTSNodeData -> Score
        reconScore parentData childData =
          let (vcp, sp) = (visitCount $ parentData, score $ parentData)
              (vcc, sc) = (visitCount $ childData, score $ childData) in
          ( sc / (fromIntegral vcc) ) +
          cExp * sqrt (  2.0*(log $ fromIntegral vcp) / (fromIntegral vcc)  )
    (Nothing, Nothing) -> EQ
    (Just _, Nothing) -> LT
    (Nothing, Just _) -> GT
\end{lstlisting}
The function splits into two cases. In the case where both children have been explored (i.e. have got MCTS data), we use expression~\ref{eq:uctnodevalue} to produce two numbers which us an ordering in the usual way. In case neither child have been explored, they are equal. In case the first has been explored but not the second, then the first is less than the second, and the final case is just the reverse of this.

With the \texttt{explore} function and \texttt{compareChildren} in hand, we can easily write the final MCTS strategy. It needs two parameters: a number of iterations and a node to work on.

\begin{minipage}{\linewidth}
\begin{lstlisting}[frame=single]
mctsStrategy 0 node = do
  return $ findBestMove node
mctsStrategy numSteps node = do
  (_, node') <- explore cExp node
  mctsStrategy (numSteps-1) node'
  where
  cExp :: Double  
  cExp = 1.0 / (sqrt 2.0)
\end{lstlisting}
\end{minipage}
The base case of zero iterations is listed first. It uses the function \texttt{findBestMove} which determines the best move for the player in turn, according to the ordering implied by \texttt{compareChildern}. In the other case, one round of \texttt{explore} is executed, and the strategy is applied recursively to the resulting node.

There are still a few minor blanks left to fill in. For the complete code, see section~\ref{sec:code}.


\section{Generating games}
\label{sec:generating_games}
\subsection{Introduction and basic definitions}

In the last section, we showed how to implement MCTS (or more specifically: UCT), which can be used directly to play games in a probabilistic manner.
The plan is to try it out on many different games.
In order to do that, we have to generate the games themselves.
The goal of this chapter is to show how to do just that.

This section talks about computationally generating positional games, given some constraints which we will cover later.
Since we will use a tool called Nauty to generate games, and since Nauty talks about graphs, it is slightly more appropriate to talk about hyper-graphs in place of positional games.

\begin{definition}
  A \emph{hypergraph} is a set of \emph{vertices}, $V$, together with a set of \emph{hyperedges} $E$, which are non-empty subsets of $V$.
\end{definition}

A positional game is equivalent to a hypergraph, if we take the so-called winning sets to be the hyperedges.

\subsection{Nauty and hypergraphs}


Nauty is a tool to generate and work with graphs, and can be downloaded at \texttt{http://cs.anu.edu.au/people/bdm/nauty/}.
We are particularly interested in the tools \texttt{genbg} and \texttt{showg}.

\texttt{genbg} is used to generate (non-isomorphic) bipartite graphs. By default it will output the graphs in the very compact g6 format.
\texttt{showg} is used to turn these g6-formatted graphs into more human-readable form.

A bipartite graph corresponds to a hypergraph in the following manner.
Let us say that the two colors of the bipartite graph are red and blue. Then we can decide that the red vertices correspond exactly to the set of vertices in our hypergraph.
We can make the blue vertices correspond to hyperedges by defining the hyperedge for a given blue vertex as those red vertices which are connected to it.


\subsection{Choosing the command line switches}
\label{sec:nautycommandline}

In this section, we work out as an example all of the hypergraphs with 2 vertices and 3 hyperedges, using Nauty.
I will assume that Nauty has been installed and your current working directory contains the executables \texttt{genbg} and \texttt{showg}.

Here is the command to generate and display all non-isomorphic bipartite graphs with 2 vertices in the first class and 3 vertices in the second class.

\begin{code}
   $ ./genbg 2 3 | ./showg
\end{code}

The output (in tabular form here, for compactness) is as follows:

\begin{tabular}{|p{1.5in} | p{1.5in} | p{1.5in} |}
\begin{minipage}{1.5in}
\begin{output}
Graph 1, order 5.
0 : ;
1 : ;
2 : ;
3 : ;
4 : ;

\end{output}
\end{minipage}
&
\begin{minipage}{1.5in}
\begin{output}
Graph 5, order 5.
0 : 3;
1 : 4;
2 : ;
3 : 0;
4 : 1;

\end{output}
\end{minipage}
& 
\begin{minipage}{1.5in}
\begin{output}
Graph 9, order 5.
0 : 2 3 4;
1 : 4;
2 : 0;
3 : 0;
4 : 0 1;

\end{output}
\end{minipage}
\\
\begin{minipage}{1.5in}
\begin{output}
Graph 2, order 5.
0 : 4;
1 : ;
2 : ;
3 : ;
4 : 0;

\end{output}
\end{minipage}
&
\begin{minipage}{1.5in}
\begin{output}
Graph 6, order 5.
0 : 3 4;
1 : 4;
2 : ;
3 : 0;
4 : 0 1;

\end{output}
\end{minipage}
&
\begin{minipage}{1.5in}
\begin{output}
Graph 10, order 5.
0 : 2 4;
1 : 3;
2 : 0;
3 : 1;
4 : 0;

\end{output}
\end{minipage}
\\
\begin{minipage}{1.5in}
\begin{output}
Graph 3, order 5.
0 : 4;
1 : 4;
2 : ;
3 : ;
4 : 0 1;

\end{output}
\end{minipage}
&
\begin{minipage}{1.5in}
\begin{output}
Graph 7, order 5.
0 : 3 4;
1 : 3 4;
2 : ;
3 : 0 1;
4 : 0 1;

\end{output}
\end{minipage}
&
\begin{minipage}{1.5in}
\begin{output}
Graph 11, order 5.
0 : 2 4;
1 : 3 4;
2 : 0;
3 : 1;
4 : 0 1;

\end{output}
\end{minipage}
\\
\begin{minipage}{1.5in}
\begin{output}
Graph 4, order 5.
0 : 3 4;
1 : ;
2 : ;
3 : 0;
4 : 0;

\end{output}
\end{minipage}
&
\begin{minipage}{1.5in}
\begin{output}
Graph 8, order 5.
0 : 2 3 4;
1 : ;
2 : 0;
3 : 0;
4 : 0;

\end{output}
\end{minipage}
&
\begin{minipage}{1.5in}
\begin{output}
Graph 12, order 5.
0 : 2 3 4;
1 : 3 4;
2 : 0;
3 : 0 1;
4 : 0 1;

\end{output}
\end{minipage}
\\
&
&
\begin{minipage}{1.5in}
\begin{output}
Graph 13, order 5.
0 : 2 3 4;
1 : 2 3 4;
2 : 0 1;
3 : 0 1;
4 : 0 1;

\end{output}
\end{minipage}
\\
\end{tabular}


Each graph is displayed as five rows -- one row for each vertex.
The rows contain the index of the vertex, followed by a list of it's neighbours.
The first two rows correspond to the two vertices in the first class and thus correspond to the vertices of our would-be hypergraphs.
The remaining three rows correspond to the vertices of the second color, and thus correspond to the would-be hyperedges.
Clearly, there are some issues to work out. Firstly, note that \texttt{Graph 1} does not give us an actual hypergraph; both hyperedges would be empty, which we do not allow. The same criticism holds for \texttt{Graph 2} and \texttt{Graph 3}.

To get around this, we use the command line switch $\texttt{-dm:n}$ where the \texttt{m} and \texttt{n} are the minimum degree of the first and second class of vertices, respectively.

With \texttt{-d0:1} we are saying that the vertices of our hypergraph may be in no hyperedges, but each hyperedge must contain at least one vertex, i.e. must not be empty.


\begin{code}
  $ ./genbg 2 3 -d0:1 | ./showg
\end{code}
The output is:

\begin{tabular}{|c|c|c|}
\begin{minipage}{1.5in}
\begin{output}
Graph 1, order 5.
  0 : 2 3 4;
  1 : ;
  2 : 0;
  3 : 0;
  4 : 0;

\end{output}
\end{minipage}

&

\begin{minipage}{1.5in}
\begin{output}
Graph 3, order 5.
  0 : 2 4;
  1 : 3;
  2 : 0;
  3 : 1;
  4 : 0;

\end{output}
\end{minipage}

&

\begin{minipage}{1.5in}
\begin{output}
Graph 5, order 5.
  0 : 2 3 4;
  1 : 3 4;
  2 : 0;
  3 : 0 1;
  4 : 0 1;

\end{output}
\end{minipage}

\\

\begin{minipage}{1.5in}
\begin{output}
Graph 2, order 5.
  0 : 2 3 4;
  1 : 4;
  2 : 0;
  3 : 0;
  4 : 0 1;

\end{output}
\end{minipage}

&

\begin{minipage}{1.5in}
\begin{output}
Graph 4, order 5.
  0 : 2 4;
  1 : 3 4;
  2 : 0;
  3 : 1;
  4 : 0 1;

\end{output}
\end{minipage}

&

\begin{minipage}{1.5in}
\begin{output}
Graph 6, order 5.
  0 : 2 3 4;
  1 : 2 3 4;
  2 : 0 1;
  3 : 0 1;
  4 : 0 1;
\end{output}
\end{minipage}

\\
\end{tabular}


There are still some issues. If we were to try to translate \texttt{Graph 1} to a hypergraph, we would get the same hyper edge three times. That is to say, we would end up with a hypergraph containing only a single hyperedge.

This issue is resolved with the command line switch \texttt{-z}, which makes sure that no two vertices in the second class can have the same neighborhood.

\begin{code}
  $ ./genbg 2 3 -z -d0:1 | ./showg
\end{code}
This yields the output:

\begin{output}
Graph 1, order 5.
  0 : 2 4;
  1 : 3 4;
  2 : 0;
  3 : 1;
  4 : 0 1;

\end{output}
So in the end we have only a single hypergraph in this class. It has two vertices, each with it's own singleton hyperedge plus a hyperedge that contains both of the vertices.
Table \ref{tab:bipartite_hypergraph_correspondence} how this bipartite graph and it's corresponding hypergraph are related.

\begin{center}
\def\arraystretch{1.0}
\begin{table}
\begin{tabular}{c c}
  \def\svgwidth{0.45\columnwidth} \input{bipartite.pdf_tex}
  &
  \def\svgwidth{0.45\columnwidth} \input{hypergraph.pdf_tex}
\end{tabular}
\caption{The bipartite graph, and it's corresponding hypergraph.}
\label{tab:bipartite_hypergraph_correspondence}
\end{table}
\end{center}

Our hypothesis at this point is that a command such as \texttt{./genbg m n -z -d0:1} will generate all non-isomorphic hypergraphs of \texttt{m} vertices and \texttt{n} edges.

More precisely, we need to prove the following results.

\begin{definition}
The \emph{neighbourhood} $\mathcal{N}(v) \subset V(G)$ of a vertex $v \in V(G)$ is defined as the set of vertices which have edges connected directly to $v$.
\end{definition}

\begin{definition}
A graph $G$ is said to be \emph{bipartite}, if there are $V_1, V_2 \subset{V(G)}$ such that $V(G) = V_1 \cup V_2$, such that $V_1 \cap V_2 = \emptyset$, and such that $\mathcal{N}(V_1) \subset V_2$ and $\mathcal{N}(V_2) \subset V_1$.
We denote these subsets by $V_1(G)$ and $V_2(G)$, and call them the \emph{first and second classes} of $G$, respectively.
\end{definition}

\begin{lemma}
Let $\mathfrak{B}$ be the set of all bipartite graphs where all vertices in the second class have distinct, non-empty neighbourhoods.
Let $\mathfrak{H}$ be the set of all hypergraphs.
Then there is a bijection, up to isomorphism, between $\mathfrak{B}$ and $\mathfrak{H}$.
\end{lemma}
\begin{proof}
Let $G \in \mathfrak{B}$. Define $\varphi(G) = (V_1(G), \mathcal{N}(V_2(G))) \in \mathfrak{H}$.
(This is a member of $\mathfrak{H}$ by hypothesis -- each element of $\mathcal{N}(V_2(G))$ is non-empty.)
In this way we have constructed a mapping from $\mathfrak{B}$ to $\mathfrak{H}$, which we claim has the desired properties.

\textbf{Injectivity:}
To show \emph{injectivity up to isomorphism}, we should show $G \cong G' \Leftrightarrow \varphi(G) \cong \varphi(G')$, for any two $G,G' \in \mathfrak{B}$.
Suppose that there is a graph isomorphism $f: G \isomarrow G'$.
Define $h: V_1(G) \rightarrow V_1(G')$ by $h(v) = f(v)$. In other words, $h$ is just $f$ restricted to $V_1(G)$.
We must now show that $h$ yields a bijection between $\mathcal{N}(V_2(G))$ and $\mathcal{N}(V_2(G'))$.
Since $\mathcal{N}(V_2(G)) \subset V_1(G)$ we get $h = f$ on $\mathcal{N}(V_2(G))$.
Now, since $f$ is an isomorphism, $f$ indeed yields a bijection between $\mathcal{N}(V_2(G))$ and $\mathcal{N}(V_2(G'))$. Therefore, so does $h$.
We have shown $G \cong G' \Rightarrow \varphi(G) \cong \varphi(G')$.
To show the converse, suppose that we have a hypergraph isomorphism between $h: \varphi(G) \isomarrow \varphi(G')$.
We define $f: G \rightarrow G'$ in two cases: $f \|_{V_1(G)} = h$ and $f \|_{V_2(G)} = \mathcal{N}^{-1} h \mathcal{N}$.
We should check that the second expression is well defined. All vertices in $V_2(G)$ have distinct neighbourhoods, which is just another way of saying that $\mathcal{N}$ yields a bijection between vertices in $V_2(G)$ and their neighbourhoods, for all $G$. We also have that $h$ forms a bijection between the neighbourhoods of $V_2(G)$ and $V_2(G')$, by it's definition. These two facts are sufficcient to show that $f \|_{V_2(G)}$ is well defined.
To show that $f$ is an isomorphism, we should also check that whenever $v_1 \sim v_2$ in $G$, we get $f(v_1) \sim f(v_2)$ in $G'$, and vice-versa.
Suppose that $v_1 \sim v_2$ in $G$, where $v_1 \in V_1(G)$ and $v_2 \in V_2(G)$. It follows that $v_1 \in \mathcal{N}(v_2)$.
Therefore $h(v_1) \in h \mathcal{N}( v_2 )$.
Now, $h(v_1) = f(v_1)$ and $h \mathcal{N} (v_2) = \mathcal{N} f (v_2) $, so we have $f(v_1) \in \mathcal{N} f (v_2)$, which implies $f(v_1) \sim f(v_2)$.
This sequence of implications also holds in reverse, and so we get $f(v_1) \sim f(v_2) \Rightarrow v_1 \sim v_2$. (Note that $h(v_1) \in h \mathcal{N}(v_2) \Rightarrow v_1 \in \mathcal{N}(v_2)$ holds because $h$ is a bijection.)

\textbf{Surjectivity:}
To show \emph{surjectivity up to isomorphism}, we should show that, given any $H \in \mathfrak{H}$, there is a $G \in \mathfrak{B}$ such that $\varphi(G) \cong H$.
So let $H = (V, \mathcal{F}) \in \mathfrak{H}$ be any hypergraph.
Clearly, we can define a set $W$ such that $| W | = | \mathcal{F} |$ and $V \cap W = \emptyset$.
Since $| W | = | \mathcal{F} |$ is finite, there exists a bijection $\phi: W \rightarrow \mathcal{F}$.
We define $G \in \mathfrak{B}$ by $V_1(G) = V$, $V_2(G) = W$, and $v \sim w \Leftrightarrow v \in \phi(w)$, where $v \in V$ and $w \in W$.
By definition, we then have $\varphi(G) = (V, \mathcal{N}(W))$.
We can also see that $\mathcal{N} \|_{W} = \phi$, and so $\mathcal{N}(W) = \mathcal{F}$.
Thus, we get $\varphi(G) = (V, \mathcal{F}) = H$.
\qed
\end{proof}

We need a bit more precision, as provided by the following result, which follow immediately from the definition of $\varphi$.

We might want to restrict the bijection on the number of vertices and hyperedges.

\begin{corollary}
Let $\mathfrak{B}_{n,m} \subset \mathfrak{B}$ be those bipartite graphs in $\mathfrak{B}$ with $n$ vertices in the first class and $m$ vertices in the second class.
Let $\mathfrak{H}_{n,m}$ and the set of all non-isomorphic hypergraphs with $n$ vertices and $m$ hyperedges.
Then there is a bijection between $\mathfrak{B}_{n,m}$ and $\mathfrak{H}_{n,m}$.
\end{corollary}

We can also restrict on the size of the hyperedges and the number of hyperedges a given vertex can ocurr in.

\begin{corollary}
Let $d,d',D,D'$ be integers.
Let $\mathfrak{B}_{d,D,d',D'} \subset \mathfrak{B}$ be those bipartite graphs $G \in \mathfrak{B}$ which satisfy $d \leq | \mathcal{N}(V_1(G)) | \leq D$ and $d' \leq | \mathcal{N}(V_2(G) | \leq D'$.
Let $\mathfrak{H}_{d,D,d',D'} \subset \mathfrak{H}$ be those hypergraphs $(V, \mathcal{F}) \in \mathfrak{B}$ with vertices that are in at least $d$ and at most $D$ hyperedges (inclusive), and $d' \leq | \mathcal{F} | \leq D'$.
Then there is a bijection between $\mathfrak{B}_{d,D,d',D'}$ and $\mathfrak{H}_{d,D,d',D'}$.
\end{corollary}

Or, we can restrict it in both ways.

\begin{corollary}
\label{cor:hypergraph_bipartite_bijection}
There is a bijection between $\mathfrak{B}_{n,m} \cap \mathfrak{B}_{d,D,d',D'}$ and $\mathfrak{H}_{n,m} \cap \mathfrak{H}_{d,D,d',D'}$.
\end{corollary}


\section{Experiments}
\label{sec:experiments}
\subsection{Introduction}

In this section we give an overview of the experiments.
We want to find out how ``good'' the UCT strategy is for various numbers of iterations.
Of course we expect that it becomes better at higher numbers of iterations, but how many are necessary?
We would also want to know at what number, or even if, it becomes virtually perfect.

We can measure the performance of UCT, for a given game, by playing with UCT at $n$ iterations as First versus perfect as Second.
Since the outcome is random, we'll want to repeat this many times, and record how many times UCT wins, looses and ties.

We also want to classify the game being played as first player win, second player win or neither player win.
As discussed in section \ref{sec:minimax}, this means running perfect vs perfect for the game and recording the outcome.


\subsection{Classes of games to study}

The number of positional games grows rapidly with the number of vertices (or positions).
If we want to study interesting games of higher number of vertices, we are going to need to focus on particular classes.

In the experiment outlined below, we will focus on games which have hyperedges (winning sets) containing either two or three vertices.
In order to limit the number of vertices in a hyperedge, we can use the arguments \texttt{-dm:n} and \texttt{-Dm:n} with \texttt{genbg}.
The argument \texttt{-dm:n} gives lower bounds, $m$ and $n$, for the minimum degrees of the first and second classes of vertices, respectively.
Recall that the first class of vertices corresponds to the vertices of the hyper-graph, and the second class corresponds to the hyperedges, as described in section \ref{sec:nautycommandline}.
Similarly, the argument \texttt{-Dm:n} specifies upper bounds on the maximum degrees for the two classes.
By corollary \ref{cor:hypergraph_bipartite_bijection}, we can list the class of hypergraphs with 4 vertices and 2 hyperedges containing either 2 or 3 vertices, and where each vertex is in at least one hyperedge, by running:

\begin{code}
  $ ./genbg -z 4 2 -d1:2 -D2:3 | ./showg
\end{code}
The output is:

\begin{tabular}{|p{1.5in} | p{1.5in} | p{1.5in} |}
\begin{minipage}{1.5in}
\begin{datalisting}
Graph 1, order 6.
  0 : 4;
  1 : 4;
  2 : 5;
  3 : 5;
  4 : 0 1;
  5 : 2 3;

\end{datalisting}
\end{minipage}
&
\begin{minipage}{1.5in}
\begin{datalisting}
Graph 2, order 6.
  0 : 4 5;
  1 : 4;
  2 : 5;
  3 : 5;
  4 : 0 1;
  5 : 0 2 3;
\end{datalisting}
\end{minipage}
&
\begin{minipage}{1.5in}
\begin{datalisting}
Graph 3, order 6.
  0 : 4 5;
  1 : 4 5;
  2 : 4;
  3 : 5;
  4 : 0 1 2;
  5 : 0 1 3;
\end{datalisting}
\end{minipage}
\\
\end{tabular}


\subsection{The experiment}
\label{sec:experiment1}

This chapter describes our experiment. Mainly the setup and contents -- results are dealt with more thoroughly in the next chapter.
We are dealing with the set of all hypergraphs with hyperedges of size either two or three, and where each vertex is in at least one hyperedge, from the following classes:

\begin{tabular}{ c | c }
\#vertices & \#hyperedges \\ \hline
3&1\ldots4 \\ \hline
4&2\ldots10 \\ \hline
5&2\ldots20 \\ \hline
6&2\ldots12 \\ \hline
\end{tabular}

That ends up being about 2.2 million hypergraphs. 2,215,838, to be exact.

For each of those hypergraphs, we run 6 tournaments of UCT versus perfect (minimax): UCT with 2,7,12,17,22 and 27 iterations, as First, versus perfect as Second.
Each of these tournaments consist of 100 games, since the outcome of a game is random. Within a given tournament, we record the number of First wins, Second wins and the number of ties.

We also classify the games as First, Second or Neither player win, by playing two optimal players against each other and recording the result.

This all ends up in a database, which is discussed in more detail in chapter \ref{sec:database_queries}.

\subsubsection{A preview of the results}

The various data gathered are viewed in several different ways in the next section, in order to get some idea about how well MCTS performs against a perfect opponent.

For instance, in section \ref{sec:results_at_large} we look at the number of wins, losses and ties at a given number of iterations of for MCTS as First versus a perfect opponent as Second. Since there are many more First-win games than Neither-win games, we look at this table separately for those two classes of games. Note that in First-win games, MCTS wins against a perfect opponent only if it manages to reproduce perfect play, and in Neither-win games MCTS will tie only if it manages to reproduce perfect play. The trend is that in First-win games, MCTS is quite weak at very low numbers of iterations, but then climbs quite rapidly toward approximately 0.96 win rate at 27 iterations. The trend in Neither-win games is different in the way that, even at extremely low numbers of iterations, we get 0.87 tie rate. The rate then proceeds to climb steadily to 0.99 at 27 iterations.

Examples of a different way to split the results up is given in section \ref{sec:more_detailed_results}.
There, the games are classified based on their number of 2-edges and 3-edges. For each such category, we study the win-rates across the number of iterations of MCTS as First versus perfect as Second.
In the first table, we quickly notice that in almost all of the cases, increasing numbers of iterations means an increasing win-rate for MCTS. There are exceptions to this in some categories (such as for 12 2-edges and 0 3-edges), but even in those cases the performance of MCTS is not bad.
This section also shows that there are different aspects to MCTS's performance. For instance, in the case of 0 2-edges and 7 3-edges, MCTS only managed to get up to a win-rate of 0.6 at most. But on the other hand, in the same category and same number of iterations, the win-rate of the perfect opponent is only 0.1, meaning that the games ended up with no winner at a rate of 0.4. This can be contrasted with, for example, the category of 2 2-edges and 10 3-edges, where on the one hand the win-rate of MCTS goes up to 0.89 at most, but at the same time, the win-rate of Second is quite high at 0.7.

\pagebreak
% bibliography
% \bibliographystyle{abbrv}
\bibliographystyle{plainnat}
\bibliography{simple}
\begin{thebibliography}{9}


\bibitem[Beck, 2008]{beck08}
József Beck, Combinatorial Games: Tic-Tac-Toe Theory
\emph{Cambridge University Press, 2008}

\bibitem[Demaine, Hearn, 2008]{demaine_hearn08}
Erik D. Demaine, Robert A. Hearn, Playing Games with Algorithms: Algorithmic Combinatorial Game Theory
\emph{http://arxiv.org/abs/cs/0106019 (arXiv:cs/0106019)}

\bibitem[MCTS Survey, 2012]{mcts_survey12}
A Survey of Monte Carlo Tree Search Methods
\emph{IEEE Transactions on AI and computational intelligence in games, Vol. 4, No 1, March 2012}

\bibitem[Winning Ways, 2001-2004]{winning_ways}
Berlekamp, Conway, Guy, Winning Ways for your Mathematical Plays
\emph{2nd edition, Wellesley, Massachusetts: A. K. Peters Ltd., 4 vols., 2001–2004;}

\bibitem[Russell, Norvig, 2012]{aimodernapproach}
Stuart Russell, Peter Norvig, Artificial Intelligence: A Modern Approach (3rd Edition) 
\emph{Prentice Hall Series in Artificial Intelligence, 2009}


\end{thebibliography}

\pagebreak
\section{Appendix A: Database queries}
\label{sec:database_queries}
In this section, we go trough, in a tutorial-like fashion, various interesting queries that one could make into the database.

It is assumed that you have installed SQLite.

\subsection{Overall database structure queries}

This section presents various ways of querying the overall structure of the database.
Feel free to experiment and combine these various queries, as long as you follow the rules of the SQL language implementation in SQLite.

\subsubsection{Query: Vertices and edges}

Suppose that we are interested in finding out, roughly, the structure of the table \texttt{mctsvsperfect} in our database.
We might first be interested in knowing which ``classes'' of hypergraphs are in the table, where by ``class'' we mean how many vertices and edges.



\begin{code}

sqlite> SELECT DISTINCT numvertices,numedges
        FROM mctsvsperfect NATURAL JOIN hypergraphs;

...

\end{code}

The output is supressed here, since it is not ordered.
If we want ordered results, we extend the previous query a little bit:


\begin{code}

sqlite> SELECT DISTINCT numvertices,numedges
        FROM mctsvsperfect NATURAL JOIN hypergraphs
        ORDER BY numvertices, numedges;

2|1
3|1
3|2
3|3
3|4
4|1
4|2
4|3
4|4
4|5
4|6
4|7
4|8
4|9
4|10
5|1
5|2
5|3
5|4
5|5
5|6
5|7
5|8
5|9
5|10
5|11
5|12
5|13
5|14
5|15
5|16
5|17
5|18
5|19
5|20
6|1
6|2
6|3
6|4
6|5
6|6
6|7
6|8
6|9
6|10
6|11
6|12
6|13

\end{code}






\pagebreak
\section{Appendix B: Code}
\label{sec:code}
\subsection {Overview}

In this section, we give an overview of the overall structure of the code and the \texttt{PoGa} library.


Some parts of the \texttt{PoGa} library, such as the implementations of the games Tic-Tac-Toe and Hex, are not relevant and are thus not brought up here.
The same goes for the entire \texttt{Graphics} submodule, which implements support for a human player for the cases of Tic-Tac-Toe and Hex.

\subsection {Positional game library}

In this section, we introduce the \texttt{PoGa} library.
One part of \texttt{PoGa} is mainly an interface used for representing and playing positional games.
However, \texttt{PoGa} also provides the two strategies discussed earlier: minimax with alpha-beta pruning, and the UCT variant of MCTS.

\texttt{PoGa} also contains implementations of arbitrarily sized ($m \times n$) Tic-Tac-Toe and Hex.

\subsubsection {Game.hs}

This module defines the \texttt{Position} type-class, which is the central interface in the \texttt{PoGa} module.
It also defines the notion of a \texttt{Strategy} and how to play two strategies against each other in \texttt{playGame}.

Note that a strategy is just a move function, but that the result of the move is a monadic position value.
If we use the \texttt{IO} monad, we can represent a human player as a strategy. If we use a monad in the \texttt{MonadRandom} type-class, we can represent a random strategy. (Though, typically not completely random, of course.)

\begin{code}
test
\end{code}

\subsubsection{SetGame.hs}

A typeclass works well as a general interface, but we need to have a concrete data-type to represent our games.

For many purposes, the following one works quite well. When it doesn't, a user of the library can always define his own data-type and use most of the functionality provided by the \texttt{PoGa} library, as long as the data-type is an instance of the \texttt{Position} typeclass, as described above.

\begin{code}
test
\end{code}

\subsubsection {Strategies.hs}

In this section, we list the strategy submodule of \texttt{PoGa}.
It contains implementations of minimax with alpha-beta pruning as well as UCT.
The implementations of these were discussed in less detail in sections \ref{sec:alpha_beta} and \ref{sec:uct}, respectively.


As mentioned, strategies take a position to a monadic position, where the monad typeically is in \texttt{MonadRandom} or is the \texttt{IO} monad.

A strategy representing a human player would require some kind of user interface, which is not provided here.
It is provided in the module named \texttt{GraphicalGame}, and the required user interfaces are specified for the cases of arbitrarily sized Tic-Tac-Toe and Hex games.

Since it is not an important part of this paper, that entire module is left out of this section for space reasons.
However, the interested reader can download them from the repository (TODO: insert repository address here).

\begin{code}
test
\end{code}


\end{document}

